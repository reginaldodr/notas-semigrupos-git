\section{Teorema de Hille-Yosida}



\begin{definition}
Seja $X$ um espaço de Banach e $A:D(A)\subset X\longrightarrow X$ um operador linear fechado. Chamamos de \dt{resolvente de $A$}, \index{Resolvente!conjunto} o seguinte conjunto
\[\rho(A)=\{\lambda \in \mathbb{C};\ \lambda \id -A \text{ é bijetor }\}\]
%O conjunto \dt{resolvente de $A$}, \index{Resolvente!conjunto} denotado por $\rho(A)$, é o conjunto de todos os $\lambda\in \mathbb{C}$ tais que:
%\begin{itemize}
%    \item $\lambda \id -A$ é bijetor;
%    \item $R(\lambda,A):=(\lambda \id -A)^{-1}: X\to $, chamado de \dt{operador resolvente} \index{Resolvente!operador} de $A$, é limitado.
%\end{itemize}
O conjunto $\sigma(A)=\mathbb{C}\setminus\rho(A)$ é dito \dt{espectro} \index{Espectro} de $A$. Para cada $\lambda\in \rho(A)$, definimos o  \dt{operador resolvente} \index{Resolvente!operador} por
\[R(\lambda,A):=(\lambda \id -A)^{-1}: X\longrightarrow D(A).\]
O qual, pelo Teorema do Gráfico fechado, é um operador linear limitado.


Note que, se $\lambda \in \rho(A)$, então para todo $x\in X$ 
\begin{equation*}
(\lambda \id -A)R(\lambda,A)x=x\Rightarrow \lambda \rla x-x=A\rla x.
\end{equation*}
Analogamente, se $x\in D(A)$, então
\begin{equation*}
R(\lambda,A)(\lambda \id -A)x=x\Rightarrow \lambda \rla x-x=\rla Ax.
\end{equation*}
Destas duas equações, podemos concluir que
\begin{equation}\label{RAAR}
A\rla x=\lambda \rla x-x=\rla Ax,\; \forall\, x\in D(A).
\end{equation}

\begin{xca}
Prove que se $\lambda, \mu\in \rho(A)$, então $\rla$ e $R(\mu,A)$ comutam.
\end{xca}

\end{definition}

\begin{theorem}[Hille-Yosida]\label{HY-contr}
Seja $X$ um espaço de Banach. Um operador linear $A:D(A)\subset X\longrightarrow X$ é o gerador infinitesimal de um 
\textbf{semigrupos de contrações} se, e somente se, 
\begin{enumerate}[$(i)$]
\item $A$ é fechado e densamente definido, i.e., $\overline{D(A)}=X$. 
\item $(0,+\infty)\subset \rho(A)$ e para todo $\lambda>0$
\begin{equation*}
\nl{R(\lambda,A)}\leq \frac{1}{\lambda}.
\end{equation*}
\end{enumerate}
\end{theorem}

A prova deste teorema será divida em  lemas.

\begin{lemma}[Condição Necessária]
Seja $A$ o gerador infinitesimal de um semigrupo de contrações. Então $A$ é fechado e densamente definido. Além disso, $(0,+\infty)\subset \rho(A)$, 
\[R(\lambda,A)x=\displaystyle\int_0^\infty e^{-\lambda \xi}S(\xi)x\, d\xi,\] 
e $\nl{R(\lambda, A)}\leq \displaystyle\frac{1}{\lambda}$.
\end{lemma}
\begin{proof} Da Proposição \ref{Afd}, temos que $A$ é fechado e densamente definido.

Dados $x\in X$ e $\lambda>0$ defina
\begin{equation*}
L_\lambda x =\int_0^\infty e^{-\lambda \xi}S(\xi)x\, d\xi.
\end{equation*}
Primeiramente, vejamos que $L_\lambda x$ está bem definida. 
Para isso, vamos usar o Teste de Weierstrass
(Proposição \ref{TWeierstrass}). Do Corolário \ref{Scontinua}, temos que a função
\begin{equation*}
(t,\lambda)\in [0,+\infty)\times (0,+\infty)\longmapsto e^{-\lambda t}S(t)x\in  X,
\end{equation*}
é contínua em $t$ para cada $\lambda\in (0,+\infty)$. 

Como $S$ é um semigrupo de contrações, temos que
\begin{equation*}
\|e^{-\lambda t}S(t)x\| =e^{-\lambda t}\|S(t)x\|\leq e^{-\lambda t}\|x\|
\leq e^{-t}\|x\| =:M(t), 
\; \forall\ (t,\lambda)\in [0,+\infty)\times (0,+\infty)
\end{equation*}
Além disso, 
\begin{align*}
 \int_0^\infty M(t)\, dt& =\int_0^\infty e^{-t}\|x\|\, dt
=\|x\| \int_0^\infty e^{- t}\, dt 
=\|x\|\lim_{b\to +\infty} \int_0^b e^{-t}\, dt\\
& =\|x\|\lim_{b\to +\infty}-e^{-t}\Big\vert_{t=0}^{t=b} =\|x\|\lim_{b\to +\infty}\left(1-e^{-b}\right)={\|x\|}<+\infty.
\end{align*}
Portanto, pelo Teste de Weierstrass, $L_\lambda$ está bem definida. Claro que $L_\lambda$ é linear e além disso, procedendo como acima, 
\begin{align*}
\left\|L_\lambda x\right\|& \leq \int_0^\infty\|e^{-\lambda t}S(t)x\|\,dt\leq
\int_0^\infty e^{-\lambda t}\|x\|\,dt\\
& = \frac{\|x\|}{\lambda}\lim_{b\to +\infty}-e^{-\lambda t}\Big\vert_{t=0}^{t=b}
=\frac{\|x\|}{\lambda},
\end{align*}
isto é, $L_\lambda$ define um operador linear limitado em $X$ e 
$\nl{L_\lambda}\leq \frac{1}{\lambda}$.

Resta mostrar que  $L_\lambda= R(\lambda,A)$, isto é, devemos mostrar que
\begin{enumerate}
\item $\forall\, x\in X,\ L_\lambda x\in D(A)$ e $(\lambda \id-A)L_\lambda x=x$, i.e., $A\left(L_\lambda x\right)=L_\lambda x -x$.
\item $\forall\, x\in D(A),$ $L_\lambda(\lambda \id-A)x=x$, i.e., $L_\lambda(Ax)=\lambda L_\lambda x-x$.
\end{enumerate}

De fato, dado $x\in X$, seja $h>0$, como $A_h\in \mathcal{L}$, temos que
\begin{align*}
A_h(L_\lambda x)& =A_h\left(\int_0^\infty e^{-\lambda \xi}S(\xi)x\, d\xi\right)
=\frac{1}{h}\int_0^\infty (S(h)-\id) e^{-\lambda \xi}S(\xi)x\, d\xi\\
&= \frac{1}{h}\int_0^\infty e^{-\lambda \xi}S(\xi+h)x\, d\xi- 
\frac{1}{h}\int_0^\infty e^{-\lambda \xi}S(\xi)x\, d\xi\\
\shortintertext{(Fazendo a mudança $s=\xi+h$ na primeira integral e trocando $\xi$ por $s$ na segunda)}
& = \frac{1}{h}\int_h^\infty e^{-\lambda (s-h)}S(s)x\, ds- 
\frac{1}{h}\int_0^\infty e^{-\lambda s}S(s)x\, ds\\
& = \frac{e^{\lambda h}}{h}\int_h^\infty e^{-\lambda s}S(s)x\, ds- 
\frac{1}{h}\int_0^\infty e^{-\lambda s}S(s)x\, ds\\
& = \frac{e^{\lambda h}}{h}\left(\int_0^\infty e^{-\lambda s}S(s)x\, ds
-\int_0^h e^{-\lambda s}S(s)x\, ds\right)- 
\frac{1}{h}\int_0^\infty e^{-\lambda s}S(s)x\, ds\\
& = \frac{e^{\lambda h}-1}{h}\int_0^\infty e^{-\lambda s}S(s)x\, ds
-\frac{e^{\lambda h}}{h}\int_0^h e^{-\lambda s}S(s)x\, ds\\
&= \frac{e^{\lambda h}-1}{h}L_\lambda x
-\frac{e^{\lambda h}}{h}\int_0^h e^{-\lambda s}S(s)x\, ds
\end{align*}
Aplicando-se o limite quando $h\to0^+$, obtemos
\begin{align*}
A\left(L_\lambda x\right)&=\lim_{h\to0^+}A_h(L_\lambda x) =\lim_{h\to0^+}\left(\frac{e^{\lambda h}-1}{h}
L_\lambda x-e^{\lambda h}\frac{1}{h}\int_0^h
e^{-\lambda s}S(s)x\, ds\right)\\
& =\frac{d}{dh}e^{\lambda h}\Big\vert_{h=0}
L_\lambda x-\lim_{h\to0^+}
\bigg(e^{\lambda h}\underbrace{\frac{1}{h}\int_0^h
\overbrace{e^{-\lambda s}S(s)x}^{\text{contínua}}\,
 ds}_{\substack{\downarrow \\ x}}\bigg)\\[-1em]
& =\lambda
L_\lambda x-x,
\end{align*}
o que prova o o item 1. Agora vamos provar o item 2. Dado $x\in D(A)$, 
\begin{align*}
L_\lambda(A)x& =\int_0^{\infty}e^{-\lambda
 \xi}S(\xi)Ax\,d\xi=\int_0^{\infty}e^{-\lambda
 \xi}\frac{d}{d\xi}S(\xi)x\,d\xi\\
& =\int_0^{\infty}\frac{d}{d\xi}\left(e^{-\lambda
 \xi}S(\xi)x\right)+\lambda e^{-\lambda
 \xi}S(\xi)x \,d\xi\\
& =\lim_{b\to +\infty}\left(e^{-\lambda
 \xi}S(\xi)x\big\vert_{\xi=0}^{\xi=b}\right)+\lambda \int_0^{\infty}e^{-\lambda
 \xi}S(\xi)x \,d\xi\\
& =\lim_{b\to +\infty}\left(e^{-\lambda
 b}S(b)x\right)-x+\lambda L_\lambda x\\
\shortintertext{(como $S$ é uma contração, $\|e^{-\lambda b}S(b)x\|\leq e^{-\lambda b}\|x\|\to 0$, quando $b\to +\infty$)}
& =-x+\lambda L_\lambda x,
\end{align*}
como queríamos.
\end{proof}

\begin{lemma} Seja $A$ satisfazendo as condições $(i)$ e $(ii)$ do Teorema \ref{HY-contr}. Então
\begin{equation*}
\lim_{\lambda \to +\infty} \lambda\rla x=x, \; \forall\,x\in X.
\end{equation*}
\end{lemma}
\begin{proof}
Se $x\in D(A)$, da identidade \eqref{RAAR}, temos que
\begin{equation*}
\|\lambda \rla x-x\|=\|\rla Ax\|\leq \frac{1}{\lambda}\|Ax\|\to 0\; 
\text{ quando } \lambda\to +\infty.
\end{equation*}
Agora, \underline{dado $x\in X$}, \underline{para cada $\ep>0$}, como $D(A)$ é denso em $X$, tome $y\in D(A)$ tal que
\[\|y-x\|< \frac{\ep}{4}.\]
E, da convergência anterior, como $y\in D(A)$,\underline{ existe $\lambda_0>0$ tal que se
$\lambda >\lambda_0$}, então
\begin{equation*}
\|\lambda \rla y-y\|< \frac{\ep}{2}.
\end{equation*}
Com isso, 
\begin{align*}
\underline{\|\lambda \rla x-x\|}& \leq \|\lambda \rla x-\lambda\rla y\|
+\|\lambda \rla y-y\|+\|y-x\|\\
& < \lambda \|\rla (x-y)\|+ \frac{\ep}{2}+\frac{\ep}{4}\\
& \leq \|x-y\|+ \frac{3\ep}{4}\\
&< \frac{\ep}{4}+\frac{3\ep}{4}=\underline{\ep}\\
\end{align*}
\end{proof}

\begin{definition}
Para cada $\lambda \in (0,+\infty) \cap \rho(A)$, definimos o operador $A_\lambda: X\longrightarrow X$, chamado \dt{aproximação de Yosida}\index{Aproximação de Yosida} de $A$, por 
\begin{equation*}
A_\lambda :=\lambda A\rla =\lambda^2\rla -\lambda \id\in \mathcal{L}(X).
\end{equation*}
\end{definition}

\begin{lemma}
 Seja $A$ satisfazendo as condições $(i)$ e $(ii)$ do Teorema \ref{HY-contr}. 
Se $A_\lambda$ é a aproximação de Yosida do operador $A$, então
\begin{equation*}
\lim_{\lambda \to +\infty}A_\lambda x=Ax,\; \forall\, x\in D(A).
\end{equation*}
\end{lemma}
\begin{proof}
Se $x\in D(A)$, então, de \eqref{RAAR} e do Lema anterior, temos que
\begin{equation*}
\lim_{\lambda \to +\infty}A_\lambda x=\lim_{\lambda \to +\infty}\lambda\rla Ax=Ax
\end{equation*}
\end{proof}

\begin{lemma}
Seja $A$ satisfazendo as condições $(i)$ e $(ii)$ do Teorema \ref{HY-contr}. Se $A_\lambda$ é  aproximação de Yosida de $A$, então $A_\lambda$ é o gerador infinitesimal de um semigrupo uniformemente contínuo de contrações $e^{tA_\lambda}$. Além disso, para cada $x\in X$, $\lambda,\mu>0$ temos que
\begin{equation*}
\left\|e^{tA_\lambda}x-e^{tA_\mu}x\right\|\leq t\|A_\lambda x-A_\mu x\|.
\end{equation*}
\end{lemma}
\begin{proof}
Como $A_\lambda \in \mathcal{L}(X)$, então $A_\lambda$ é o gerador infinitesimal do semigrupo uniformemente contínuo $e^{tA_\lambda}$. Além disso,
\begin{multline*}
\left\|e^{tA_\lambda}\right\|
 =\left\|e^{t\lambda^2\rla-t\lambda \id}\right\|
\leq \left\|e^{t\lambda^2\rla}\right\|\left\|e^{-t\lambda \id} \right\|
=e^{-t\lambda }\left\|e^{t\lambda^2\rla}\right\|\\ 
\leq e^{-t\lambda }e^{t\lambda^2\left\|\rla\right\|}
 \leq e^{t\lambda }e^{-t\lambda }=1
\end{multline*}
portanto, um semigrupo de contrações.

Das definições de $A_\lambda$, $A_\mu$, $e^{tA_\lambda}$ e $e^{tA_\mu}$ 
comutam. Como $A_\lambda,  A_\mu\in\mathcal{L}(X)$, defina 
\[f(s)=e^{tsA_\lambda}e^{t(1-s)A_\mu}x=e^{tA_\mu}e^{st(A_\lambda-A_\mu)} \]
e note que
\begin{enumerate}
\item $f(1)=e^{tA_\lambda}x$, $f(0)=e^{tA_\mu}x$
\item $f'(s)=t(A_\lambda-A_\mu)e^{tsA_\lambda}e^{t(1-s)A_\mu}x$
\end{enumerate}
Consequentemente, pelo Teorema Fundamental do Cálculo,
\begin{align*}
\left\|e^{tA_\lambda}x-e^{tA_\mu}x \right\|
& =\left\|f(1)-f(0) \right\|=
\left\|\int_0^1 f'(s)\,ds \right\|\\
& =\left\|\int_0^1 
t(A_\lambda-A_\mu)e^{tsA_\lambda}e^{t(1-s)A_\mu}x\right\|\\
& \leq t\underbrace{\left\|e^{tsA_\lambda}e^{t(1-s)A_\mu}\right\|}_{\leq 1}   \left\|A_\lambda x-A_\mu x\right\|\\
& \leq t \left\|A_\lambda x-A_\mu x\right\|
\end{align*}
\end{proof}










