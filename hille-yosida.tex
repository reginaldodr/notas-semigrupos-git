\section{Teorema de Hille-Yosida}



\begin{definition}
Seja $X$ um espaço de Banach e $A:D(A)\subset X\longrightarrow X$ um operador linear fechado. Chamamos de \dt{resolvente de $A$}, \index{Resolvente!conjunto} o seguinte conjunto
\[\rho(A)=\{\lambda \in \mathbb{C};\ \lambda \id -A \text{ é bijetor }\}\]
%O conjunto \dt{resolvente de $A$}, \index{Resolvente!conjunto} denotado por $\rho(A)$, é o conjunto de todos os $\lambda\in \mathbb{C}$ tais que:
%\begin{itemize}
%    \item $\lambda \id -A$ é bijetor;
%    \item $R(\lambda,A):=(\lambda \id -A)^{-1}: X\to $, chamado de \dt{operador resolvente} \index{Resolvente!operador} de $A$, é limitado.
%\end{itemize}
O conjunto $\sigma(A)=\mathbb{C}\setminus\rho(A)$ é dito \dt{espectro} \index{Espectro} de $A$. Para cada $\lambda\in \rho(A)$, definimos o  \dt{operador resolvente} \index{Resolvente!operador} por
\[R(\lambda,A):=(\lambda \id -A)^{-1}: X\longrightarrow D(A).\]
O qual, pelo Teorema do Gráfico fechado, é um operador linear limitado. 

\end{definition}

\begin{theorem}[Hille-Yosida]
Seja $X$ um espaço de Banach. Um operador linear $A:D(A)\subset X\longrightarrow X$ é o gerador infinitesimal de um \underline{semigrupos de contrações} se, e somente se, 
\begin{enumerate}[$(i)$]
\item $A$ é fechado e densamente definido, i.e., $\overline{D(A)}=X$. 
\item $(0,+\infty)\subset \rho(A)$ e para todo $\lambda>0$
\begin{equation*}
\nl{R(\lambda,A)}\leq \frac{1}{\lambda}.
\end{equation*}
\end{enumerate}
\end{theorem}

A prova deste teorema será divida em  lemas.

\begin{lemma}[Condição Necessária]
Seja $A$ o gerador infinitesimal de um semigrupo de contrações. Então $A$ é fechado e densamente definido. Além disso, $(0,+\infty)\subset \rho(A)$, 
\[R(\lambda,A)x=\displaystyle\int_0^\infty e^{-\lambda \xi}S(\xi)x\, d\xi,\] 
e $\nl{R(\lambda, A)}\leq \displaystyle\frac{1}{\lambda}$.
\end{lemma}
\begin{proof} Da Proposição \ref{Afd}, temos que $A$ é fechado e densamente definido.

Dados $x\in X$ e $\lambda>0$ defina
\begin{equation*}
L_\lambda x =\int_0^\infty e^{-\lambda \xi}S(\xi)x\, d\xi.
\end{equation*}
Primeiramente, vejamos que $L_\lambda x$ está bem definida. 
Para isso, vamos usar o Teste de Weierstrass
(Proposição \ref{TWeierstrass}). Do Corolário \ref{Scontinua}, temos que a função
\begin{equation*}
(t,\lambda)\in [0,+\infty)\times (0,+\infty)\longmapsto e^{-\lambda t}S(t)x\in  X,
\end{equation*}
é contínua em $t$ para cada $\lambda\in (0,+\infty)$. 

Como $S$ é um semigrupo de contrações, temos que
\begin{equation*}
\|e^{-\lambda t}S(t)x\| =e^{-\lambda t}\|S(t)x\|\leq e^{-\lambda t}\|x\|
\leq e^{-t}\|x\| =:M(t), 
\; \forall\ (t,\lambda)\in [0,+\infty)\times (0,+\infty)
\end{equation*}
Além disso, 
\begin{align*}
 \int_0^\infty M(t)\, dt& =\int_0^\infty e^{-t}\|x\|\, dt
=\|x\| \int_0^\infty e^{- t}\, dt 
=\|x\|\lim_{b\to +\infty} \int_0^b e^{-t}\, dt\\
& =\|x\|\lim_{b\to +\infty}-e^{-t}\Big\vert_{t=0}^{t=b} =\|x\|\lim_{b\to +\infty}\left(1-e^{-b}\right)={\|x\|}<+\infty.
\end{align*}
Portanto, pelo Teste de Weierstrass, $L_\lambda$ está bem definida. Claro que $L_\lambda$ é linear e além disso, procedendo como acima, 
\begin{align*}
\left\|L_\lambda x\right\|& \leq \int_0^\infty\|e^{-\lambda t}S(t)x\|\,dt\leq
\int_0^\infty e^{-\lambda t}\|x\|\,dt\\
& = \frac{\|x\|}{\lambda}\lim_{b\to +\infty}-e^{-\lambda t}\Big\vert_{t=0}^{t=b}
=\frac{\|x\|}{\lambda},
\end{align*}
isto é, $L_\lambda$ define um operador linear limitado em $X$ e 
$\nl{L_\lambda}\leq \frac{1}{\lambda}$.

Resta mostrar que  $L_\lambda= R(\lambda,A)$, isto é, devemos mostrar que
\begin{enumerate}
\item $\forall\, x\in X,\ L_\lambda x\in D(A)$ e $(\lambda \id-A)L_\lambda x=x$, i.e., $A\left(L_\lambda x\right)=L_\lambda x -x$.
\item $\forall\, x\in D(A),$ $L_\lambda(\lambda \id-A)x=x$, i.e., $L_\lambda(Ax)=\lambda L_\lambda x-x$.
\end{enumerate}

De fato, dado $x\in X$, seja $h>0$, como $A_h\in \mathcal{L}$, temos que
\begin{align*}
A_h(L_\lambda x)& =A_h\left(\int_0^\infty e^{-\lambda \xi}S(\xi)x\, d\xi\right)
=\frac{1}{h}\int_0^\infty (S(h)-\id) e^{-\lambda \xi}S(\xi)x\, d\xi\\
&= \frac{1}{h}\int_0^\infty e^{-\lambda \xi}S(\xi+h)x\, d\xi- 
\frac{1}{h}\int_0^\infty e^{-\lambda \xi}S(\xi)x\, d\xi\\
\shortintertext{(Fazendo a mudança $s=\xi+h$ na primeira integral e trocando $\xi$ por $s$ na segunda)}
& = \frac{1}{h}\int_h^\infty e^{-\lambda (s-h)}S(s)x\, ds- 
\frac{1}{h}\int_0^\infty e^{-\lambda s}S(s)x\, ds\\
& = \frac{e^{\lambda h}}{h}\int_h^\infty e^{-\lambda s}S(s)x\, ds- 
\frac{1}{h}\int_0^\infty e^{-\lambda s}S(s)x\, ds\\
& = \frac{e^{\lambda h}}{h}\left(\int_0^\infty e^{-\lambda s}S(s)x\, ds
-\int_0^h e^{-\lambda s}S(s)x\, ds\right)- 
\frac{1}{h}\int_0^\infty e^{-\lambda s}S(s)x\, ds\\
& = \frac{e^{\lambda h}-1}{h}\int_0^\infty e^{-\lambda s}S(s)x\, ds
-\frac{e^{\lambda h}}{h}\int_0^h e^{-\lambda s}S(s)x\, ds\\
&= \frac{e^{\lambda h}-1}{h}L_\lambda x
-\frac{e^{\lambda h}}{h}\int_0^h e^{-\lambda s}S(s)x\, ds
\end{align*}
Com isso, 
\begin{align*}
A\left(L_\lambda x\right)&=\lim_{h\to0^+}A_h(L_\lambda x) =\lim_{h\to0^+}\left(\frac{e^{\lambda h}-1}{h}
L_\lambda x-e^{\lambda h}\frac{1}{h}\int_0^h
e^{-\lambda s}S(s)x\, ds\right)\\
& =\frac{d}{dh}e^{\lambda h}\Big\vert_{h=0}
L_\lambda x-\lim_{h\to0^+}\left(e^{\lambda h}\underbrace{\frac{1}{h}\int_0^h
\overbrace{e^{-\lambda s}S(s)x}^{\text{contínua}}\, ds}_{\to\; x}\right)\\
& =\lambda
L_\lambda x-x,
\end{align*}
o que prova o o item 1.





\end{proof}