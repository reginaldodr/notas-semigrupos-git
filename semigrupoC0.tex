\chapter{Semigrupos de Classe $C^0$}

\section{Semigrupos}



\begin{definition}
     Seja $X$ um espaço de Banach. Dizemos que a aplicação $S:[0,+\infty)\longrightarrow \mathcal{L}(X)$ é um  \dt{semigrupo de operadores limitados em X} quando:
\begin{enumerate}
    \item $S(0)=\operatorname{Id}$;
    \item $S(t+s)=S(t)S(s)$, $\forall t,s\in [0,+\infty)$;

Dizemos que $S$ é \dt{de classe $C^0$ ou fortemente contínuo} se

    \item $\lim\limits_{t\to0^+} \|(S(t)-\operatorname{Id})x\|=0$, $\forall x\in X$.

Dizemos que $S$ é \dt{uniformemente contínuo} se 

\item $\lim\limits_{t\to 0^+}\|S(t)-\operatorname{Id}\|_{\mathcal{L}(X)}=0$.
\end{enumerate}
\end{definition}

   \begin{example} São exemplos de semigrupos:
        \begin{enumerate}
            \item Sejam $X$ um espaço de Banach e  $A\in \mathcal{L}(X)$. Define-se a aplicação exponencial por
            \[e^{A}=\operatorname{Id}+\sum_{n=1}^\infty \frac{A^n}{n!}.\]
            Pode-se mostrar que esta série é absolutamente convergente e define $e^{A}\in \mathcal{L}(X)$, veja \cite[Apêndice 2]{gomes1985semigrupos}. Neste caso,  $e^{tA}:[0,+\infty)\to \mathcal{L}(X)$, quando $A\in \mathcal{L}(X)$, é um semigrupo.
            \item Seja $X=C_b(\R)$  o espaço das funções $f:\R\longrightarrow \R$ uniformemente contínuas e limitadas, com a norma do $\sup$. Então $S(t)f(s)=f(t+s)$ definie um semigrupo de classe $C^0$.
        \end{enumerate}
    \end{example}


\begin{proposition}
    Se $S$ é um semigrupo de classe $C^0$ em X, então existem $\mu\geq 0$ e $M\geq 1$ tais que 
    \begin{equation}\label{des.wT}
    \|S(t)\|_{\mathcal{L}(X)}\leq Me^{\mu t},\ \forall t\geq 0.  
    \end{equation}    
    Em particular, $\|S(t)\|_{\mathcal{L}(X)}$ é uma função limitada em todo intervalo $[0,T]$. 
\end{proposition}

\begin{proof}
Vamos aplicar o Teorema de Banach-Steinhaus (Teorema \ref{th-BS} )  à família  $\{S(t)\}_{t\geq 0}$ em $\mathcal{L}(X)$, onde $X$ é Banach. Para isso, basta mostrar que esta família é pontualmente limitada. 

De fato,  $S$ é um semigrupo de classe $C^0$, então
\[\lim\limits_{t\to 0^+}\|S(t)x-x\|=0,\ \forall x\in X.\]
Com isso, dado $x\in X$, para $\ep=1$, existe $\delta>0$ tal que, se $0\leq t\leq \delta$, então
\[\|S(t)x\|\leq \|S(t)x-x\|+\|x\|\leq 1+\|x\|=C_x.\]

Portanto, pelo Teorema de Banach-Steinhaus, $\{S(t)\}_{0\leq t\leq\delta}$ é uniformemente limitada, isto é, $\exists\, M>0$ tal que
\begin{equation}\label{eq1}
\|S(t)\|_{\mathcal{L}(X)}\leq M,\, \text{ para todo  } t\in [0,\delta]. 
\end{equation}
Além disso, $M\geq \|S(0)\|_{\mathcal{L}(X)}=\|\operatorname{Id}\|_{\mathcal{L}(X)}=1$.

Por outro lado, dado $t>\delta$, pelo algorítimo da divisão, existem $n\in\mathbb{N}$ e $r\in [0,\delta)$ tais que $t=n\delta+r$. Com isso, do item 2 da definição de Semigrupo e da desigualdade \eqref{eq1}, temos que
\begin{align*}
 \|S(t)\|_{\mathcal{L}(X)} &
 \leq \|S(n\delta +r)\|_{\mathcal{L}(X)}= \|S(n\delta)S(r)\|_{\mathcal{L}(X)}=
 \|\overbrace{S(\delta)\cdots S(\delta)}^{n \text{ vezes}}S(r)\|_{\mathcal{L}(X)}\\
& \leq 
\|S(\delta)\|^n_{\mathcal{L}(X)}\|S(r)\|_{\mathcal{L}(X)}\leq M^{n+1}. 
\end{align*}
 Note que, como $n\leq t/\delta$ e $M\geq 1$, temos que $M^n\leq M^{t/\delta}$. Da desigualdade anterior,
\[\|S(t)\|_{\mathcal{L}(X)}\leq M M^{t/\delta}=Me^{\frac{t}{\delta}\log(M)}
=Me^{\mu t},\]
onde $\mu=\frac{1}{\delta}\log(M)$.

Em particular, 
\[\|S(t)\|_{\mathcal{L}(X)}\leq Me^{\mu t}\leq Me^{\mu T},\, \forall t\in [0,T].\]
Em outras palavras, $\nl{S(t)}$ é limitada em $[0,T]$.
\end{proof}


\begin{corollary}\label{Scontinua}
Todo semigrupo de classe $C^0$ é fortemente contínuo em $[0,+\infty)$, i.e., 
para todo $x\in X$, $S(\cdot)x\in C^0([0,+\infty);X)$. Em outras palavras,
\begin{center}
$t\in [0,+\infty)\longmapsto S(\cdot)x\in  X$ é contínua.
\end{center}
\end{corollary}

\begin{proof}
    Dado $x\in X$, devemos mostrar que, para todo $t\in [0,+\infty)$,
    \[\|S(t+h)x-S(t)x\|\to 0,\, \text{ quando } h\to 0.\]
Do item (3) da definição de semigrupos, já temos que
  \[\|S(h)x-S(0)x\|=\|(S(h)-\operatorname{Id})x\|\to 0,\, \text{ quando } h\to 0.\]
Isto é, $S(\cdot)x$ é contínua em $t=0$. Dado $t\in (0,+\infty)$, se $h>0$, então
\begin{align*}
    \|S(t+h)x-S(t)x\|& =\|S(t)S(h)x-S(t)x\|=\|S(t)(S(h)-\operatorname{Id})x\|\\
    & \leq \|S(t)\|_{\mathcal{L}(X)}\|(S(h)-\operatorname{Id})x\|\\
    & \leq Me^{\mu t}\|(S(h)-\operatorname{Id})x\|\to 0,\, \text{ quando } h\to 0^+.
\end{align*}

Se $h<0$, então seja $k=-h>0$. Daí, se $h\to 0^+$, então $k\to0^+$. Com isso,
\begin{align*}
    \|S(t+h)x-S(t)x\|& = \|S(t-k)x-S(t)x\|=\|S(t-k)x-S(t-k+k)x\|\\
    & =\|S(t-k)x-S(t-k)S(k)x\|=\|S(t-k)(\id-S(k)x\|\\
    & \leq \|S(t-k)\|_{\mathcal{L}(X)}\|(S(k)-\operatorname{Id})x\|\\
    & \leq Me^{\mu t}\|(S(k)-\operatorname{Id})x\|\to 0,\, \text{ quando } k\to 0^+.
\end{align*}
\end{proof}

Vamos melhorar a estimativa \eqref{des.wT} através do seguinte teorema. 
\begin{theorem}\label{th2.5}
Seja $S$ um semigrupo de classe $C^0$ em $X$. Então,
\[\lim\limits_{t\to +\infty}\frac{\log(\nl{S(t)})}{t}=\inf\limits_{t>0}\frac{\log(\nl{S(t)})}{t}=:\omega_0\]
e para cada $\omega>\omega_0$, existe {$M\geq 1$} tal que 
\begin{equation}\label{Sbound}
{\nl{S(t)}\leq Me^{\omega t},\, \forall t\geq 0.}
\end{equation}
\end{theorem} 


\begin{remark}
Quando $\omega_0<0$, então para $\omega=0$, temos que 
\[\nl{S(t)}\leq M,\ \forall t\geq 0. \]
Neste caso, dizemos que $S$ é um \dt{semigrupo uniformemente limitado}. Se, além disso, $M=1$, $S$ é dito \dt{semigrupo das contrações}. \index{semigrupo das contrações}
    
\end{remark}


\begin{lemma}\label{lem2.5}
    Seja $p:[0,+\infty)\longrightarrow \R$ uma função {subaditiva}, isto é, $p(t+s)\leq p(t)+p(s)$. Se $p$ é limitada superiormente em todo intervalo limitado, então $p(t)/t$ tem um limite quanto $t\to +\infty$ e 
    \[\lim\limits_{t\to +\infty}\frac{p(t)}{t}=\inf\limits_{t>0}\frac{p(t)}{t}.\]
Prova: Ver \cite[Lema 1.2.5]{gomes1985semigrupos} 
\end{lemma}


\begin{proof}[Prova do Teorema \ref{th2.5}]
Primeiramente, vejamos que $p(t)=\log\left(\nl{S(t)}\right)$ é subaditiva. De fato,
como $S(t)\in \mathcal{L}(X)$, temos que
\[\nl{S(t+s)}\leq \nl{S(t)S(s)}\leq \nl{S(t)}\nl{S(s)},\, \forall t,s\geq 0.\]
Assim, como a função $\log$ é crescente, temos que
\begin{align*}
p(t+s)& =\log\left(\nl{S(t+s)}\right)=\log\left(\nl{S(t)}\nl{S(s)}\right)\\
& \leq\log\left(\nl{S(t)}\right)+\log\left(\nl{S(s)}\right)\\
& \leq p(t)+p(s).
\end{align*}

A fim de aplicarmos o lema anterior, resta mostrar que $p$ é limitada superimente em todo intervalo limitado. Com efeito, seja $(a,b)$ um intervalo limitado em $[0,+\infty)$. Em particular, $(a,b)\subset [0,b]$. Portanto, da desigualdade \eqref{des.wT}, temos que
\begin{equation*}
    \nl{S(t)}\leq Me^{\mu b},\, \forall t\in (a,b).
\end{equation*}
Isto é, $\nl{S(t)}$ é limitada superiormente em todo intervalo limitado. Como $\log$ é crescente, temos que  $p$ também o é. Logo, do Lema \ref{lem2.5}, temos que
\begin{equation*}
    \lim\limits_{t\to \infty}\frac{\log\left(\nl{S(t)}\right)}{t}
    =\inf\limits_{t>0}\frac{\log\left(\nl{S(t)}\right)}{t}=:\omega_0.
\end{equation*}
Se $\omega>\omega_0$, tome $\ep=\omega-\omega_0$, pela definição de limite, existe $t_0>0$ tal que se $t>t_0$, então
\[\frac{\log\left(\nl{S(t)}\right)}{t}-\omega_0<\ep=\omega-\omega_0.\]
Donde, 
\[\frac{\log\left(\nl{S(t)}\right)}{t}<\omega\Rightarrow 
\nl{S(t)} \leq e^{\omega t},\, \forall t>t_0.\]
Por outro lado, da desigualdade \eqref{des.wT}, temos que
\[
\nl{S(t)} \leq Me^{\mu t_0}=:M_0,\, \forall t\in [0,t_0].
\]
E como $S(0)=\id$, então $M_0\geq 1$.

\noindent\textbf{1º caso:} $\omega\geq 0$.\\
Vimos que
\begin{equation*}
    \begin{cases}
    \log\left(\nl{S(t)}\right)\leq \log(M_0), & 0\leq t\leq t_0,\\
        \log\left(\nl{S(t)}\right)< t\omega, & t>t_0.        
    \end{cases}
\end{equation*}
Com isso, 
\begin{equation*}
 \log\left(\nl{S(t)}\right)\leq \max\{ \log(M_0),t\omega\}\leq \log(M_0)+t\omega ,\, \forall t\in [0,+\infty).
\end{equation*}
Donde,
\begin{equation*}
    \nl{S(t)}\leq e^{\log(M_0)+t\omega}\leq M_0e^{t\omega},\, \forall t\in [0,+\infty). 
\end{equation*}

\noindent\textbf{2º caso:} $\omega< 0$.\\
Neste caso, se $t>t_0$, como $-t_0\omega\geq 0$ e $\log(M_0)\geq 0$, então
\begin{equation*}
\log\left(\nl{S(t)}\right)\leq t\omega<\underbrace{\log(M_0)-t_0\omega}_{\geq 0}+t\omega, \, \forall t>t_0. 
\end{equation*}

Por outro lado, se $t\leq t_0$, então $t\omega-t_0\omega\geq 0$, daí,
\begin{equation*}
\log\left(\nl{S(t)}\right)\leq \log(M_0)<\log(M_0)\underbrace{-t_0\omega+t\omega}_{\geq 0}, \, \forall 0\leq t\leq t_0. 
\end{equation*}

Em resumo,
\begin{equation*}
\log\left(\nl{S(t)}\right)\leq \log(M_0){-t_0\omega+t\omega}, \, \forall t\geq 0. 
\end{equation*}
Consequentemente,
\begin{equation*}
    \nl{S(t)}\leq \underbrace{M_0e^{-t_0\omega}}_{M}e^{t\omega}=Me^{t\omega}, \forall t\geq 0.
\end{equation*}
\end{proof}

\begin{definition}
    Seja $S$ um semigrupo de classe $C^0$ em $X$. O \dt{gerador infinitesimal}\index{Gerador Infinitesimal} de $S$ é o operador $A:D(A)\subset X\longrightarrow X$ definido por 
\[D(A)=\left\{x\in X;\ \lim\limits_{h\to 0^+} \frac{S(h)x-x}{h} \text{ existe}\right\},\]
\[Ax:=\lim\limits_{h\to 0^+} \frac{S(h)x-x}{h},\, x\in D(A)\]
\end{definition}

Dado $S$ um semigrupo de classe $C^0$ em $X$, vamos designar por $A_h$ o operador linear limitado 
    \[A_hx=\frac{S(h)x-x}{h},\ \forall x\in X.\]

\begin{proposition}
     $D(A)$ é um subespaço vetorial de $X$ e $A$ é um operador linear.
\end{proposition}
\begin{proof}
    Exercício.
\end{proof}


\begin{remark}
De acordo com a definição acima, todo semigrupo tem um gerador infinitesimal associado. A principal questão é a recíproca, isto é, {\color{blue} quando um operador $A:D(A)\subset X\longrightarrow X$ é o gerador infinitesimal de algum semigrupo?} O Teorema de Hille-Yosida nos dará as condições para responder a essa pergunta.
\end{remark}


%A seguir, apresentamos algumas notações que serão adotadas.
%\begin{enumerate}
% \item Dado $S$ é um semigrupo de classe $C^0$ em $X$, vamos designar por $A_h$ o operador linear limitado 
%    \[A_hx=\frac{S(h)x-x}{h},\ \forall x\in X.\]
%\item Escrevemos $S(t)=e^{tA}$ para dizer que $A$ é o gerador infinitesimal de um semigrupo de classe $C^0$ em $X$.
%\item Escrevemos $A\in G(M,\omega)$ para exprimir que $A$ é o gerador infinitesimal de um semigrupo de classe $C^0$, que satisfaz a condição:
%\[\|e^{tA}\|\leq {M}e^{{\omega} t}, \forall t\geq 0.\]
%\end{enumerate}

\begin{theorem}
Seja $S$ um semigrupo de classe $C^0$ em $X$ e $A$ seu gerador infinitesimal. Dado $x\in D(A)$, então 
\[S(t)A x\in C^0([0,+\infty);D(A))\cap C^1([0,+\infty);X)\]
e
\[\frac{d}{dt}\left(S(t)x\right)=AS(t)x=S(t)Ax.\]
\end{theorem}
\begin{proof}\noindent


\noindent\textbf{Afirmação 1:} Se $x\in D(A)$, então $S(t)x\in D(A)$ e $A(S(t)x)=S(t)Ax$.

Dado $x\in D(A)$, seja $y=S(t)x$. Primeiramente, vamos mostrar que
$\lim\limits_{h\to 0^+} A_hy=S(t)Ax$. Para isso, note que
\begin{align*}
    A_hy&=\frac{S(h)y-y}{h}=\frac{S(h)S(t)x-S(t)x}{h}=\frac{S(t+h)x-S(t)x}{h}\\
    &=\frac{S(t)S(h)x-S(t)x}{h}=S(t)\frac{S(h)x-x}{h}=S(t)A_hx.
\end{align*}
Como $x\in D(A)$, temos que $\lim\limits_{h\to^0}A_hx=Ax$. Além disso, $S(t)\in \mathcal{L}(X)$, então $S(t)$ é contínua. Portanto, 
\[
\lim\limits_{h\to 0^+} A_hy=\lim\limits_{h\to 0^+} S(t)A_hx
=S(t)\left(\lim\limits_{h\to 0^+} A_hx\right)=S(t)Ax.
\]

Neste caso, provamos que $S(t)x=y\in D(A)$ e que $Ay=S(t)Ax$, ou seja, $A(S(t)x)=S(t)Ax$.

\noindent\textbf{Afirmação 2:} $\frac{d}{dt}S(t)x=A(S(t)x)=S(t)Ax,\ \forall x\in D(A).$

Primeiramente, note que
\begin{equation*}
\begin{split}
    A(S(t)x)& =\lim\limits_{h\to 0^+} A_h(S(t)x) =\lim_{h \to 0^+} \frac{S(h)S(t)x - S(t)x}{h}\\
    &= \lim_{h \to 0^+} \frac{S(t+h)x - S(t)x}{h} = \frac{d^+}{dt} S(t)x, \quad \forall x \in D(A).
\end{split}
\end{equation*}
Com isso, temos que 
    \begin{equation}\label{d+}
\frac{d}{dt}^+ S(t)x = A(S(t)x) = S(t)(Ax).    
\end{equation}

Agora, vamos calcular a derivada pela esquerda.
\[
\frac{d}{dt}^- S(t)x = \lim_{\delta \to 0^-} \frac{S(\overbrace{t+\delta}^{>0})x - S(t)x}{\delta}, \ \text{ para } -t < \delta < 0.
\]
Fazendo $\delta=-h$, temos que $0<h<t$ e 
\begin{equation}\label{eq2.3}
\begin{split}
    \frac{d}{dt}^- S(t)x 
    & = \lim_{h \to 0^+} \frac{S(t-h)x - S(t)x}{-h}=
    \lim_{h \to 0^+} \frac{S(t-h)x - S(t-h)S(h)x}{-h}\\
    &=  \lim_{h \to 0^-} S(t-h)\left(\frac{x - S(h)x}{-h}\right)
    =\lim_{h \to 0^+} S(t-h)A_hx\\
    &= \lim_{h \to 0^+} \Big(S(t-h)(A_hx-Ax)+S(t-h)Ax\Big).
\end{split}
\end{equation}
Do Corolário \eqref{Scontinua}, temos que $f(h)=S(t-h)Ax$ é contínua em $[0,t)$, portanto
\begin{equation}\label{eq2.4}
\lim_{h \to 0^+} S(t-h)Ax=\lim_{h \to 0^+}f(h)=f(0)=S(t)Ax.
\end{equation}
Por outro lado, do Teorema \ref{th2.5}, temos que
\[\nl{S(t-h)}\leq Me^{\omega(t-h)}\leq Me^{\omega t},\ \forall h\in [0,t),\]
donde,
\begin{equation}\label{eq2.5}
\begin{split}
 \lim_{h \to 0^+}\|S(t-h)(A_hx-Ax)\|
 & \leq \lim_{h \to 0^+}\nl{S(t-h)}\|A_hx-Ax\|\\
 & \leq \lim_{h \to 0^+}Me^{\omega t}\|A_hx-Ax\|=0.
\end{split}
\end{equation}
Com isso, \eqref{eq2.3}, \eqref{eq2.4} e \eqref{eq2.5} implicam que 
\begin{equation}\label{d-}
\frac{d}{dt}^-S(t)x=S(t)Ax=A(S(t)x),\ \forall x\in D(A).    
\end{equation}
Portanto, de \eqref{d+} e \eqref{d-}, temos que
\begin{equation*}
\frac{d}{dt}S(t)x=S(t)Ax=A(S(t)x),\ \forall x\in D(A).    
\end{equation*}
\end{proof}


\begin{example} \textbf{Existência e Unicidade de um PVI}

Seja $S$ um semigrupo de classe $C^0$ em $X$ e $A$ seu gerador infinitesimal. Se {$x_0\in D(A)$}, então $x(t)=S(t)x_0$ define {uma única solução} do PVI
\[
\begin{cases}
    x'(t)=Ax,\ t\in [0,+\infty)\\
    x(0)=x_0.
\end{cases}
\]
Se {$x_0\not\in D(A)$} em $X$, então $x(t)=S(t)x_0$ {\color{red}não é diferenciável}. Neste caso, dizemos que $x=x(t)$ é uma \dt{solução branda} (\dt{mild solution}, em inglês.)\index{Solução!branda} \index{Solução!fraca} do PVI.
\end{example}

\begin{xca}
Seja $S$ um semigrupo de classe $C^0$ em $X$ e $A$ seu gerador infinitesimal. Se $x\in D(A)$ mostre que
\begin{equation}\label{TFC1}
S(t)x-S(s)x=\int_s^t AS(\tau)x\,d\tau =\int_s^t S(\tau)Ax\,d\tau
\end{equation}
\end{xca}

\begin{proposition}\label{prop2.11}
Seja $S$ um semigrupo de classe $C^0$ em $X$ e $A$ seu gerador infinitesimal. Então para todo $x\in X$, 
\[\int_0^tS(s)x\,ds\in D(A)\ \text{ e }\ A\left(\int_0^t S(s)x\,ds\right)=S(t)x-x.\]
\end{proposition}
\begin{proof}
   Dado $t\in (0,+\infty)$, seja 
\begin{equation*}
v=\int_0^t S(s)x\, ds.
\end{equation*}
Basta mostar que $\lim\limits_{h\to 0^+} A_hv=S(t)x-x$, pois da definição de gerador infinitesimal, teremos que
\[v\in D(A) \text{ e } Av=S(t)x-x.\]

Note que, como $A_h \in \mathcal{L}(X)$, da Proposição \ref{KthA3.2}, temos que
\begin{align*}
A_hv& =A_h\left(\int_0^t S(s)x\,ds\right)=\int_0^t A_h(S(s)x)\,ds\\
& =\int_0^t \frac{S(h)S(s)x-S(s)x}{h}\,ds
= \frac{1}{h}\int_0^t S(h+s)x-S(s)x\,ds\\
&= \frac{1}{h}\int_0^t S(h+s)x\,ds- \frac{1}{h}\int_0^t S(s)x\,ds\\
\shortintertext{(Mudandaça de variáveis $\tau=h+s$ na primeira integral e fazendo $s=\tau$ na segunda)}
&=\frac{1}{h}\int_h^{t+h} S(\tau)x\,d\tau- \frac{1}{h}\int_0^t S(\tau)x\,d\tau\\
&=\left(\frac{1}{h}\int_h^{t} S(\tau)x\,d\tau+\frac{1}{h}\int_t^{t+h} S(\tau)x\,d\tau\right) -
 \left(\frac{1}{h}\int_0^h S(\tau)x\,d\tau+\frac{1}{h}\int_h^t S(\tau)x\,d\tau\right)\\
&=\frac{1}{h}\int_t^{t+h} S(\tau)x\,d\tau-\frac{1}{h}\int_0^h S(\tau)x\,d\tau.
\end{align*}
Como $S(\cdot)x$ é contínua (Corolário \ref{Scontinua}), pela identidade \eqref{ineq.VM}, 
\begin{equation*}
\lim\limits_{h\to 0^+} A_hv =\lim\limits_{h\to 0^+} \left(\frac{1}{h}\int_t^{t+h} S(\tau)x\,d\tau-\frac{1}{h}\int_0^h S(\tau)x\,d\tau\right) S(t)x-S(0)x=S(t)x-x.
\end{equation*}
Como queríamos.
\end{proof}

\begin{proposition}
Seja $S$ um semigrupo de classe $C^0$ em $X$ e $A$ seu gerador infinitesimal. Então $A$ é fechado e seu domínio é denso em $X$.
\end{proposition}
\begin{proof}
\noindent

\noindent\textbf{1. $D(A)$ é denso em $X$.}

Dado $x\in X$, basta mostrar que 
\[v_h=\frac{1}{h}\int_0^h S(t)x\,dt\in D(A) \text{ e } v_h\to x, \text{ quando } h\to 0^+.\]
De fato, que $v_h\in D(A)$ decorre diretamente da Proposição \ref{prop2.11}. Como $S(\cdot)x$ é contínua, pela identidade \eqref{ineq.VM}, 
\[\lim\limits_{h\to 0^+} v_h=\lim\limits_{h\to 0^+}\frac{1}{h}\int_0^h S(t)x\,dt=x.\]

\noindent\textbf{1. $A$ é fechado.}

Seja $(x_n)_{n\in \mathbb{N}}\subset D(A)$ tal que $x_n\to x$ e $Ax_n\to y$ 
em  $X$. Devemos mostrar que $x\in D(A)$ e $Ax=y$.



Como $A_h$ é contínuo, da identidade \eqref{TFC1}, temos que

\begin{equation}\label{aux.inq1}
A_hx =\lim\limits_{n\to +\infty}A_hx_n
=\lim\limits_{n\to +\infty}\frac{1}{h}\left(S(h)x_n-x_n\right)  =\lim\limits_{n\to +\infty}\frac{1}{h}\int_0^h S(t)Ax_n\,dt
\end{equation}
Da desigualdade \eqref{Sbound}, temos que
\begin{equation*}
\|S(t)Ax_n-S(t)y\|\leq \nl{S(t)}\|Ax_n-y\|\leq Me^{\omega t} \|Ax_n-y\|\leq Me^{\omega h}\|Ax_n-y\|.
\end{equation*}
Donde, 
\begin{align*}
\left|\frac{1}{h}\int_0^h S(t)Ax_n-S(t)y\,dt\right|& \leq 
\frac{1}{h}\int_0^h\|S(t)Ax_n-S(t)y\|\,dt  \leq \frac{1}{h}\int_0^h \underbrace{Me^{\omega h}\|Ax_n-y\|}_{\text{não depende de $t$}}\,dt \\
& \leq Me^{\omega h}\|Ax_n-y\|\to 0, \text{ quando } n\to +\infty. 
\end{align*}
Da da identidade \eqref{aux.inq1}, temos que
\begin{equation*}
A_hx=\lim\limits_{n\to +\infty}\frac{1}{h}\int_0^h S(t)Ax_n\,dt=
\frac{1}{h}\int_0^h S(t)y\,dt.
\end{equation*}
Por fim, como $x\in D(A)$ e $S(\cdot)y$ é contínua (Corolário \ref*{Scontinua}), da identidade \eqref{ineq.VM}, temos que
\[
Ax=\lim_{h \to 0^+} A_hx=\lim_{h \to 0^+} \frac{1}{h}\int_0^h S(t)y\,dt=S(0)y=y.
\]
\end{proof}

\begin{remark}
    Esta última proposição nos dá uma condição necessária para que um operador $A$ seja o gerador infinitesimal de um semigrupo.
\end{remark}

\begin{proposition}[Unicidade]
Sejam $S_1,S_2$ dois semigrupos de classe $C^0$ em $X$ e $A_1,A_2$ os respectivos geradores infinitesimais. Então $S_1=S_2$.
\end{proposition}
\begin{proof}
    {\color{red} ESCREVER}
\end{proof}

\begin{definition}
    Seja $S(t)=e^{tA}$ em $X$. Defina $A^0=\operatorname{Id}$, $A^1=A$ e, supondo que $A^{n-1}$ esteja definido, vamos definir $A^n$ pondo:
\[D(A^n)=\{x\in D(A^{n-1});\ A^{n-1}x\in D(A)\},\]
\[A^nx=A(A^{n-1}x), \ \forall x\in D(A^{n}).\]
\end{definition}

\begin{proposition}
Seja $S(t)=e^{tA}$ em $X$. Temos:

\begin{enumerate}[(i)]
 \item $\mathcal{D}\left(A^n\right)$ é um subespaço de $X$ e $A^n$ é um operador linear de $X$;

 \item Se $x \in \mathcal{D}\left(A^n\right)$, então $S(t) x \in \mathcal{D}\left(A^n\right), \forall t \geq 0 \mathrm{e}$
\[
\frac{d^n}{d t^n} S(t) x=A^n S(t) x=S(t) A^n x, \forall n \in \mathbb{N}
\]
\item É válida a fórmula de Taylor: se $x \in \mathcal{D}\left(A^n\right)$, então
\[
S(t) x=\sum_{k=0}^{n-1} \frac{(t-a)^k}{k!} A^k S(a) x+\frac{1}{(n-1)!} \int_a^t(t-\tau)^{n-1} A^n S(\tau) x d \tau
\]
\item  $(S(t)-I)^n x=\int_0^t \cdots \int_0^t S\left(\tau_1+\cdots+\tau_n\right) A^n x\, d \tau_1 \cdots d \tau_n, \forall x \in \mathcal{D}\left(A^n\right)$;
\item $\displaystyle\bigcap_n \mathcal{D}\left(A^n\right)$ é denso em $X$.
\end{enumerate}
\end{proposition}

\begin{proof}
    {\color{red} ESCREVER}
\end{proof}