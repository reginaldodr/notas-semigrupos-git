%\chapter{Preliminares}

\chapter{Noções Básicas sobre Espaços Normados}

\section{Espaços Normados}
No decorrer do presente capítulo, denotaremos por $\mathbb K$ o corpo dos números reais ou o corpo dos números complexos.
%No decorrer do presente capítulo, denotaremos por $\mathbb K$ o corpo dos números reais ou o corpo dos números complexos. Além disso, quando $(M,d)$ for um espaço métrico, cada bola aberta, cada bola fechada e cada esfera em $M$ serão denotadas por
%\[
%\displaystyle B(a;r)=\{y\in M; d(y,a)<r\},
%\]
%
%\[
%\displaystyle B(a;r)=\{y\in M; d(y,a)\leq r\}
%\]
%e
%\[
%\displaystyle B(a;r)=\{y\in M; d(y,a)=r\},
%\]
%respectivamente, onde $a\in M$ e $r>0$.

\begin{definition}\label{norma}
    Seja $X$ um $\mathbb K$-espaço vetorial. Uma aplicação
    \[
    \displaystyle \| \cdot \|: X\longrightarrow \R
    \]
    é dita uma \dt{norma} \index{norma} se, para quaisquer $x,y\in X$ e $\lambda \in \mathbb K$, as seguintes condições se verificarem:
    \begin{enumerate}[(a)]
    \item $\|x\|\geq 0$;
    \item Se $\|x\|=0$, então $x=0$;
    \item $\|\lambda x\|=|\lambda| \|x\|$;
    \item $\|x+y\|\leq \|x\|+\|y\|$.
    \end{enumerate}
    Nesse caso, o par $(X,\|\cdot\|)$ é dito um \dt{espaço normado} \index{Espaço!normado}.
\end{definition}

\begin{remark}
Em um espaço normado $(X,\|\cdot\|)$, valem:
    \begin{itemize}
    \item[(a)] $\|0\|=0$;
    \item[(b)] $|\|x\|-\|y\|| \leq \|x-y\|$ para quaisquer $x,y \in X$.
    \end{itemize}
\end{remark}

\begin{example}\label{ex21}
    Dado um número inteiro positivo $n$, não é difícil verificar que
    \[
    \displaystyle \|\cdot \|_0 : (x_1,\cdots , x_n) \in \mathbb K^n \longmapsto \bigg( \sum_{j=1}^{n} |x_j|^2\bigg)^{1/2} \in \R ,
    \]
    \[
    \displaystyle \|\cdot \|_1 : (x_1,\cdots , x_n) \in \mathbb K^n \longmapsto \max_{1\leq j \leq n} |x_j| \in \R 
    \]
    e
    \[
    \displaystyle \|\cdot \|_2 : (x_1,\cdots , x_n) \in \mathbb K^n \longmapsto \sum_{j=1}^{n} |x_j| \in \R 
    \]
    são normas em $\mathbb K ^n$.
\end{example}

\begin{definition}
Seja $A$ um conjunto não vazio. Dizemos que $f:A\longrightarrow \R$ é uma \dt{função limitada em $\mathbb{K}$} (ou simplismente limitada) quando  quando existir uma constante  $M>0$ tal que $|f(x)|\leq M$ para todo $x\in A$.
\end{definition}

\begin{example}\label{limitadas}
Seja $A$ um conjunto não vazio. Denotemos por $\mathcal B (A)$ o conjunto de todas as funções limitadas $f: A\longrightarrow \mathbb K$. Dados $f,g \in \mathcal B (A)$ e $\lambda \in \mathbb K$, definamos
\begin{itemize}
\item $(f+g)(x):=f(x)+g(x)$; 
\item $(\lambda f)(x) = \lambda f(x)$,
para cada $x\in A$.
\end{itemize}
Com as operações operações de adição e multiplicação por escalar, pontualmente dadas acima, $\mathcal B (A)$ é um $\mathbb K$-espaço vetorial. Além disso, não é difícil constatar que
\[
\displaystyle f\in \mathcal B (A) \longmapsto \| f\| = \sup_{x\in A} |f(x)| \in \R
\]
é uma norma em $\mathcal B (A)$. Observemos ainda que a convergência de sequências em $\mathcal B (A)$ é exatamente a noção de convergência uniforme. 
\end{example}

\begin{definition}
    Com as notações do Exemplo \ref{limitadas},
    \[
    \ell_{\infty} := \mathcal B (\mathbb N),
    \]
    que é o espaço normado de todas as sequências limitadas cujos termos pertencem a $\mathbb K$.
\end{definition}


\begin{definition}
Sejam $X$ e $Y$ dois espaços normados.
\begin{itemize}
\item[(a)] Dizemos que uma sequência $(x_n)_{n=a}^{\infty}$ em $X$ converge para $a\in X$ se, para cada $\varepsilon >0$, existir $n_0 \in \mathbb N$ tal que:
\[
\displaystyle n\in \mathbb N \text{ e } n\geq n_0 \Longrightarrow \|x_n - a\|_X < \varepsilon .
\]
\item[(b)] Dizemos que uma função $f:X\longrightarrow Y$ é contínua em $a\in X$ se, para cada $\varepsilon >0$, existir $\delta >0$ tal que
\[
x\in X \emph{ e } \|x-a\|_X <\delta \Longrightarrow \|f(x) - f(a)\|_Y < \varepsilon .
\]
\end{itemize}
{\color{red} definições de convergência e continuidade}
\end{definition}



\begin{example}\label{ex23}
    Sendo $X$ um espaço normado, denotemos por $\mathcal C_b (X)$ o subconjunto de $\mathcal B (X)$ formado por todas as funções contínuas e limitadas de $X$ em $\mathbb K$. Não é difícil constatar que $\mathcal C_b (X)$ é um subespaço vetorial fechado de $\mathcal B (X)$.
\end{example}

O próximo resultado traz importantes caracterizações das aplicações lineares contínuas.

\begin{proposition}\label{continuous}
Sejam $X$ e $Y$ dois espaços normados, e consideremos uma aplicação linear $T:X\longrightarrow Y$. As seguintes condições são equivalentes:

\begin{itemize}
\item[(a)] $T$ é uniformemente contínua;
\item[(b)] $T$ é contínua;
\item[(c)] $T$ é contínua em $0\in X$;
\item[(d)] Existe $C>0$ tal que $\|T(x)\|_Y \leq C \|x\|_X$ para todo $x\in X$.
\end{itemize}
\end{proposition}

\begin{proof}
É claro que $(a)\Longrightarrow (b) \Longrightarrow (c)$. Para vermos que $(c)\Longrightarrow (d)$, , tomemos $\varepsilon =1$. Como $T$ é contínua em $0\in X$, existe $\delta >0$ tal que
\[
\displaystyle \|T(x)\|_Y =\|T(x) - T(0)\|_Y <\varepsilon =1,
\]
sempre que $x\in X$ e $\|x\|_X = \|x-0\|_X <\delta$. Assim, para todo $w\in X\setminus \{0\}$, temos
\[
\displaystyle \bigg\|T\bigg( \frac{\delta}{2} \bigg( \frac{w}{\|w\|_X} \bigg)\bigg) \bigg\|_Y <1.
\]
Logo, 
\[
\|T(w)\|_Y \leq \frac{2}{\delta}  \|w\|_X \text{  para todo  } w\in X,
\]
inclusive se $w=0$.

Agora, para vermos que $(d)\Longrightarrow (a)$, basta observarmos que
\[
\|T(x) - T(y)\|_Y =\|T(x-y)\|_Y \leq C \|x-y\|_X
\]
para quaisquer $x,y\in X$. Ou seja: $T$ é uma aplicação lipschitziana e, portanto, uniformentente contínua.
\end{proof}

\begin{definition}
    Sejam $X$ e $Y$ dois espaços normados. Denotaremos por $\mathcal L (X,Y)$ o espaço vetorial de todas as transformações lineares e contínuas de $X$ em $Y$, com as operações de adição e multiplicação por escalar definidas pontualmente. Cabe observar que:
    \begin{enumerate}[(a)]
    \item Quando $X=Y$, escreveremos $\mathcal L (X)$ em vez de $\mathcal L (X,X)$;
    \item Quando $Y=\mathbb K$, denotaremos $\mathcal L (X,\mathbb K)$ por $X'$, que é conhecido como \dt{o dual topológico} \index{dual topológico} de X;
    \item O conjunto de todos os funcionais lineares $\varphi : X\longrightarrow \mathbb K$, contínuos ou não, será denotador por $X^{\ast}$, que é conhecido como \dt{o dual algébrico} \index{dual algébrico} de X.
    \end{enumerate}
\end{definition}

\begin{remark}
Seja $X$ um espaço normado. É sempre verdade que $X' \subset X^{\ast}$. Além disso, quando $X$ tem dimensão infinita, sempre temos $X'\neq X^{\ast}$ (veja a Proposição \ref{dual-topVSalg}).
\end{remark}

\begin{remark}
Sejam $X$ e $Y$ dois espaços normados. Não é difícil constatar que:
\begin{itemize}
\item[(a)] Se $X$ tem dimensão finita, então toda aplicação linear $T:X\longrightarrow Y$ é contínua;
\item[(b)] Se X tem dimensão infinita, sempre existe uma aplicação linear $T:X\longrightarrow Y$ que não é contínua (veja a Observação \ref{opilimitado}).
\end{itemize}
\end{remark}

\begin{definition}\label{bounded}
Sejam $X$ e $Y$ dois espaços normados. Uma aplicação linear $T:X 
\longrightarrow Y$ é dita \dt{limitada} \index{Operador Linear!limtado} 
se 
\begin{equation}\label{ltdo}
\sup_{x\in X\setminus \{0\}} \frac{\|T(x)\|_Y}{\|x\|_X} 
<\infty.
\end{equation}

\end{definition}

\begin{remark}
Sejam $X$ e $Y$ dois espaços normados. Em virtude da Proposição 
\ref{continuous}, uma aplicação linear $T:X\longrightarrow Y$ é contínua 
se, e somente se, é limitada. 
\end{remark}

\begin{example}
Sejam $X$ e $Y$ dois espaços normados. Vejamos que 
\[
T\in \mathcal L (X,Y) \longmapsto \|T\|_{\mathcal L (X,Y)} = \sup_{x\in X\setminus \{0\}} \frac{\|T(x)\|_Y}{\|x\|_X} \in \R
\]
é uma norma em $\mathcal L (X,Y)$. De fato, as condições (a), (b) e (c) da Definição \ref{norma} são claras. Além disso, se $T,S\in \mathcal L (X,Y)$, temos
\begin{equation*}
\displaystyle \|(T+S)(x)\|_Y=\|T(x)+S(x)\|_Y \leq \|T(x)\|_Y +\|S(x)\|_Y \leq (\|T\|_{\mathcal L (X,Y)} + \|S\|_{\mathcal L (X,Y)})\|x\|_X
\end{equation*}
para todo $x\in E$, ou seja,
\begin{equation*}
    \displaystyle \|T+S\|_{\mathcal L (X,Y)}=\sup_{x\in X\setminus \{0\}} \frac{\|(T+S)(x)\|_Y}{\|x\|_X} \leq \|T\|_{\mathcal L (X,Y)}+\|S\|_{\mathcal L (X,Y)}.
\end{equation*}
Assim, temos realmente uma norma em $\mathcal L (X,Y)$.
\end{example}

\begin{definition}
Sejam $X$ e $Y$ dois espaços normados. Uma aplicação linear $T:X\longrightarrow Y$ é dita um \dt{isomorfismo topológico} \index{isomorfismo topológico} se for um homeomorfismo. 
\end{definition}

\begin{definition}
Sejam $\|\cdot\|_1$ e $\|\cdot\|_2$ duas normas em um mesmo espaço vetorial $X$. Dizemos que tais normas são \dt{equivalentes} \index{norma!equivalentes} se a aplicação identidade
\[
\displaystyle I_X : (X,\|\cdot\|_1) \longrightarrow (X,\|\cdot\|_2)
\]
for um isomorfismo topológico.
\end{definition}

\begin{corollary}
Sejam $\|\cdot\|_1$ e $\|\cdot\|_2$ duas normas em um mesmo espaço vetorial $X$. As seguintes condições são equivalentes:
\begin{itemize}
\item[(a)] $\|\cdot\|_1$ e $\|\cdot\|_2$ são equivalentes;
\item[(b)] Existem constantes $A>0$ e $B>0$ tais que
\[
A\|x\|_2 \leq \|x\|_1 \leq \|x\|_2
\]
para todo $x\in X$.
\end{itemize}
\end{corollary}

\begin{proof}
Basta aplicar a Proposição \ref{continuous}.
\end{proof}

\section{Espaços de Banach}

\begin{definition}
Seja $X$ um espaço normado. Uma sequência $(x_n)_{n\in \mathbb N}$ em $X$ é dita ser uma \dt{sequência de Cauchy} \index{Sequência de Cauchy} quando para todo $\ep>0$ dado, existe um número $n_0\in \mathbb N$ tal que 
\begin{equation*}
    m,n\in \mathbb N \emph{ e } m,n>n_0 \Longrightarrow \|x_m-x_n\|< \ep.
\end{equation*}
Dizemos que $X$ é um \dt{espaço de Banach} \index{Espaço! de Banach} se ele for completo, isto é, quando toda sequência de Cauchy for  convergente.
\end{definition}

\begin{remark}\label{subespaço}
Se $X$ é um {\color{orange}espaço de Banach}, então cada subespaço fechado $M$ de $X$ é também um espaço de Banach, com a norma induzida pela norma de $X$.
\end{remark}

\begin{example}
Dado um inteiro positivo $n$, $\mathbb K ^n$ é um espaço de Banach com a norma $\|\cdot\|_0$ do Exemplo \ref{ex21}. Na verdade, como as normas $\| \cdot\|_0$, $\| \cdot\|_1$ e $\| \cdot\|_2$ são duas a duas equivalentes, $(\mathbb K ^n ,\| \cdot\|_j )$ é um espaço de Banach para todo $j\in \{1,2,3\}$. 
\end{example}

\begin{example}
Dado um conjunto $A\neq \emptyset$, não  é difícil constatar que $\mathcal B (A)$, introduzido no Exemplo \ref{limitadas}, é um espaço de Banach. Em particular, $\ell _{\infty}$ é um espaço de Banach.
\end{example}



\begin{remark}\label{propbanach}
    Aqui enfatizamos algumas observações importantes sobre espaços de Banach:
\begin{enumerate}[(a)]
\item Sendo $X$ e $Y$ dois espaços normados, para que $\mathcal L (X,Y)$ seja um espaço de Banach, é necessário e suficiente que $Y$ seja completo.
%\item[(b)] Para cada $p\in [1,\infty )$, $\ell_p$ é um espaço de Banach.
\item Para que um espaço normado $X$ seja completo, é necessário e suficiente que toda série absolutamente convergente em $X$ seja convergente.
\end{enumerate}
\end{remark}

{\color{red}VIRAR UMA DEFINIÇÃO}
\begin{definition}
Sejam $X$ e $Y$ serão dois espaços de Banach, e $A:D(A)\longrightarrow Y$ um operador linear, não necessariamentre limitado (veja a Definição \ref{bounded}), onde $D(A)$ é um subespaço vetorial de $X$. Dizemos que $A$ está \textbf{desamente definido} se $\overline{D(A)}=X$.
\end{definition}

\section{A exponencial de um operador}

Iniciamos esta seção relembrando que a função logaritmo natural é a bijeção contínua (com inversa contínua), definida por
\[
\displaystyle \log : x\in (0,+\infty ) \longmapsto \int_{1}^{x} \frac{dt}{t} \in \mathbb \R.
\]
A inversa de $\log :(0,+\infty ) \longrightarrow \mathbb R$ é a função exponencial $\exp :\mathbb R \longrightarrow (0,+\infty)$, e usualmente escrevemos 
\[
\displaystyle e^{x} :=\exp(x) \text{  para cada } x\in \mathbb R.
\]
Cabe recordar que:
\begin{itemize}
\item $e^{0}=1$;
\item $e^{x+y} = e^x \cdot e^{y}$ para quaisquer $x,y\in \mathbb R$;
\item $(e^x)' = e^x$ para todo $x\in \mathbb R$. 
\end{itemize}
Para uma construção detalhada e propriedades dessas funções, veja \cite[Cap. 6]{figueiredo1996analise}

De tais propriedades, dados $a\in \mathbb R$ e $x_0 \in \mathbb R$, podemos constatar que a única função $x:[0,+\infty) \longrightarrow \mathbb R$ solucionando o \textbf{problema de valor inicial}
\[
\begin{cases}
    x'(t)=ax,\ t\in [0,+\infty);\\
    x(0)=x_0,
\end{cases}
\]
é dada por $x(t)=x_0 e^{at}$. 

\begin{remark}\label{baby}
Denotando $x=x(t)$ por $S(t)x_0$, fazemos as seguintes considerações:

\begin{enumerate}[(a)]
\item Para cada $t\in \R$, 
\[
\displaystyle S(t): x_0 \in \R \longmapsto S(t)x_0 \in \R
\]
é uma função linear. Realmente, dados $x_0, y_0, c\in \R$, consideremos $x(\cdot) = S(\cdot)x_0$ e $y(\cdot)=S(\cdot)y_0$, que são as soluções de
\[
\begin{cases}
    x'=ax,\ \emph{ em } [0,+\infty);\\
    x(0)=x_0,
\end{cases}
\]
e
\[
\begin{cases}
    y'=ay,\ \emph{ em } [0,+\infty);\\
    y(0)=y_0,
\end{cases}
\]
respectivamente. Nesse caso, 
\[
\displaystyle z(\cdot):=cx(\cdot)+y(\cdot)=cS(\cdot)x_0 + S(\cdot )y_0
\]
soluciona 
\[
\begin{cases}
    z'=az,\ \emph{ em } [0,+\infty);\\
    z(0)=cx_0+y_0,
\end{cases}
\]
que possui uma única solução. Em outras palavras, $z(\cdot) = S(\cdot)(cx_0+y_0)$, isto é, fixado $t\in [0,+\infty)$,
\[
\displaystyle S(t)(cx_0+y_0)=cS(t)x_0+S(t)y_0.
\]
\item $S(0)x_0 = x(0)=x_0$, para cada $x_0$ fixado em $\mathbb R$, ou seja, $S(0)$ é exatamente a função identidade $I:x\in \R \longmapsto x\in \R$;
\item Fixados $t,s\in [0,+\infty)$ e $x_0 \in \R$, temos
\begin{equation*}
\displaystyle S(t+s)x_0 = x(t+s)= x_0 e^{a(t+s)} = [x_0 e^{as}] e^{at} = x(s) e^{at} =S(t)x(s) = S(t) S(s)x_0.
\end{equation*}
Daí, $S(t+s)$ e $S(t)S(s)$ são funções lineares idênticas de $\R$ em $\R$. Observemos também que $y=y(\cdot)=S(\cdot)x(s)$ é a única solução de 
\[
\begin{cases}
    y'=ay,\ \text{ em } [0,+\infty);\\
    y(0)=x(s).
\end{cases}
\]
\end{enumerate}
\end{remark}

À luz da Observação \ref{baby}, é comum que nas disciplinas voltadas para o estudo das Equações Diferenciais Ordinárias sejam abordados (quantitativa e/ou qualitativamente) os problemas de valor inicial da forma
\[
\begin{cases}
    X'(t)=AX,\ t\in [0,+\infty);\\
    X(0)=X_0 \in \mathcal M_{n\times 1} (\R ),
\end{cases}
\]
onde $A$ é uma matriz fixa em $\mathcal M _{n\times n} (\mathbb R )$, e a solução procurada é um caminho 
\[
X: [0,+\infty) \longrightarrow \mathcal M _{n\times 1} (\R ).
\]
Em analogia ao caso unidimensional, é conhecido que as soluções do contexto matricial são unicamente dadas por
\[
X(t)=e^{tA}X_0 \text{ para todo } t\in [0,+\infty ).
\]
Dito isso, relembramos abaixo o conceito de \textbf{exponencial de uma matriz}. Para tanto, podemos considerar 
\[
\|A\|=\bigg( \sum_{i,j=1}^{n} |a_{ij}|^2 \bigg)^{1/2}
\]
para cada $A=[a_{ij}]_{i,j=1}^{n}$, que é uma norma em $\mathcal M_{n\times n} (\R)$.
\begin{definition}\label{expmatrix}
    Seja $A\in \mathcal M_{n\times n} (\R)$. A exponencial da $A$ é dada por
    \begin{equation}\label{exp}
    \displaystyle e^{A}:=\sum_{j=0}^{\infty} \frac{A^j}{j!}. 
    \end{equation}
\end{definition}
A seguir, enfatizamos alguns fatos cruciais:
\begin{remark}\label{property}
    \begin{enumerate}[(a)]
    \item A Definição \ref{expmatrix}  está bem posta, pois a série de matrizes, dada em \eqref{exp}, é absolutamente convergente para qualquer $A\in \mathcal M _{n\times n} (\R)$, ou seja, 
    \[
    \displaystyle \sum_{j=0}^{\infty} \frac{\| A^j \|}{j!}
    \]
    é sempre uma série de números reais convergente.
    \item Fixados $A\in \mathcal M_{n\times n} (\R)$ e $X_0 \in \mathcal M_{n\times 1} (\R)$, temos
    \[
    \displaystyle \frac{d}{dt}(e^{tA}X_0) = Ae^{tA}X_0.
    \]
    É sabido que $S(t)X_0 := e^{tA} X_0$ a única solução do problema de valor inicial
    \[
\begin{cases}
    X'(t)=AX,\ t\in [0,+\infty);\\
    X(0)=X_0 \in \mathcal M_{n\times 1} (\R ),
\end{cases}
\]
    \item A aplicação $S:[0,+\infty) \longrightarrow \mathcal L (\mathcal M_{n\times 1} (\R))$ possui propriedades análogas àquelas apresentadas na Observação \ref{baby}, dedicada ao caso unidimensional. Mais precisamente,
    \begin{itemize}
    \item Para cada $t\in [0,+\infty)$, $S(t)\in \mathcal L (\mathcal M_{n\times 1} (\R))$;
    \item $S(0):\mathcal M_{n\times 1} (\R) \longrightarrow \mathcal M_{n\times 1} (\R)$ é o operador identidade;
    \item Para quaisquer $t,s\in [0,+\infty)$, vale $S(t+s)=S(t)S(s)$ (composição de funções). Esta propriedade não é imediata (veja \cite{rirsch1974differential} para maiores detalhes). 
    \end{itemize}
    \item Do ponto de vista quantitativo, a resolução de sistemas de equações diferenciais lineares (caso matricial supracitado) passa por identificar a matriz na forma canônica de Jordan que esteja na mesma classe de semelhança de $A$. Também para um tratamento qualitativo, a análise espectral de $A$ é uma estratégia eficaz no que diz respeito ao comportamento das soluções do sistema (veja \cite{rirsch1974differential}). 
    \item Todas as considerações feitas sobre a exponencial de uma matriz $A\in \mathcal M_{n\times n} (\R)$ podem ser feitas para a exponencial de um operador linear $T\in \mathcal L (\R ^n)$, o que permite resolver, de forma análoga, um sistema da forma 
\[
\begin{cases}
    x'(t)=T(x),\ t\in [0,+\infty);\\
    x(0)=x_0 \in \R ^n,
\end{cases}
\]
para o qual a solução 
\[
x(t)=(x_1 (t), \cdots , x_n (t))
\]
é um caminho em $\R^n$ dado por $e^{tT}x_0$ para todo $t\in [0,+\infty)$.

    \end{enumerate}
\end{remark}
Tendo em vista a Observação \ref{property}(e), parece natural pensar sobre a resolução e a análise do sistema de equações diferenciais
\[
\begin{cases}
    \mathbf{x'}(t)=T(\mathbf{x}),\ t\in [0,+\infty);\\
    \mathbf{x}(0)=\mathbf{x_0} \in X,
\end{cases}
\]
onde $X$ é um espaço de Banach e $T\in \mathcal L (X)$.

\begin{definition}\label{expoperator}
    Sejam $X$ um espaço de Banach e $T\in \mathcal L (X)$. A exponencial da $T$ é dada por
    \begin{equation}\label{expt}
    \displaystyle e^{T}:=\sum_{j=0}^{\infty} \frac{T^j}{j!},
    \end{equation}
    onde $T^0:=I$ e $T^{n+1} := T\circ T^{n}$ para todo inteiro $n\geq 0$.
\end{definition}

\begin{remark}
Considerando as notações da Definição \ref{expoperator}, não é difícil constatar que a série que define $e^{T}$ converge absolutamente.  Como $X$ é um espaço de Banach, $\mathcal L(X)$ também o é (relembre a Observação \ref{propbanach}(a)). Assim, pela Observação \ref{propbanach}(b), $e^{T}\in \mathcal{L}(X)$ encontra-se bem definida. Esta observação será retomada no próximo capítulo (veja o Exemplo \ref{ex}), já explorando a noção de semigrupo uniformemente contínuo. Um pouco mais adiante, nos Teoremas \ref{th-der} e \ref{PVI}, concluiremos que $\mathbf{x} (t)=e^{tT} \mathbf{x_0}$ é a única solução de
\[
\begin{cases}
    \mathbf{x'}(t)=T(\mathbf{x}),\ t\in [0,+\infty);\\
    \mathbf{x}(0)=\mathbf{x_0} \in X.
\end{cases}
\]
\end{remark}

No presente texto introdutório, o principal objetivo é obter uma condição necessária e suficiente para que o problema
\[
\begin{cases}
    \mathbf{x'}(t)=A(\mathbf{x}),\ t\in [0,+\infty);\\
    \mathbf{x}(0)=\mathbf{x_0} \in X,
\end{cases}
\]
possua solução, onde $X$ é um espaço de Banach e $A:D(A)\longrightarrow X$ é uma aplicação linear definida em um subespaço vetorial $D(A)$ de $X$. Mais precisamente, isto consistirá em demonstrar o importante Teorema \ref{HY-contr} de Hille-Yosida, que representa um marco muito importante da teoria geral dos semigrupos.
%%%%%%%%%%%%%%%%%%%%%%%%%%%%%%%%%%%%%%%%%%%%%%%%%%%%%%%%%%%%%%%%%%%%%%%%%%%%%%


%\section{Operadores Lineares Ilimitados}

%Sejam $X$ e $Y$ dois espaços de Banach. Seja $A:D(A)\subset X\longrightarrow Y$ um operador linear,  onde $D(A)$ é um subespaço de $X$, chamado de \dt{domínio}\index{Operador Linear!domínio} de $A$.
%Dizemos que $A$ é \dt{limitado (ou contínuo)}  se $D(A)=X$ e se existe  $C>0$ tal que 
%\begin{equation}\label{ltdo}
%\|Ax\|_{Y}\leq C\|x\|_X, \ \forall x\in X.
%\end{equation}
%Denotaremos por $\mathcal{L}(X,Y)$ o espaço de Banach dos \dt{operadores lineares limitados} com norma dada por
%\[\|A\|_{_{\mathcal{L}(X,Y)}}=\sup\limits_{x\in X, x\neq 0}\frac{\|Ax\|_{_Y}}{\|x\|_{_X}},\]
%por $\mathcal{L}(X)=\mathcal{L}(X,X)$ e $X'=\mathcal{L}(X,\R)$.
%$A$ é dito ser \dt{ilimitado} \index{Operador Linear!ilimitado} quando não satisfazer \eqref{ltdo}. Dizemos que $A$ é \dt{densamente definito} \index{Operador Linear!desamente definido} se $\overline{D(A)}=X$.

%\begin{theorem}[Banach-Steinhaus]\label{th-BS}
%	Sejam $X$ um espaço de Banach e $Y$ um espaço vetorial normado. Se $\{T_i\}_{i\in I}$  é uma família (não necessariamente enumerável) em $\mathcal{L}(X,Y)$ pontualmente limitada, então $\{T_i\}_{i\in I}$ é uniformemente limitada, isto é, se 
%	 \[\sup_{i\in I}\|T_ix\|< \infty,\ \forall x\in X,\] 
%	 então
%	 \[\sup_{i\in I}\|T_i\|_{\mathcal{L}(X,Y)}<\infty.\]
%\end{theorem}

\section{Integrais Vetoriais}

\begin{definition}
Sejam $X$ um espaço de Banach e $u:[a,b]\longrightarrow X$ uma aplicação tal que, para cada $\varphi\in X'$,  a função real
\[t\in [a,b] \longmapsto \langle\varphi, u(t) \rangle_{X'X}\in \R,\]
seja integrável. Dizemos que \dt{$u$ é integrável} se existe um vetor $v\in X$ que satisfaz:
\[\langle \varphi, v\rangle_{X',X}=\int_a^b \langle\varphi, u(t) \rangle_{X'X}\,dt,\ \forall \varphi\in X'.\]
Em caso afirmativo, $v$ é único e escrevemos
\[v=\int_a^b u(t)\,dt.\]
\end{definition}

\begin{proposition}\label{KthA3.2}
Se $u:[a,b]\longrightarrow X$ é {contínua}, então $u$ é integrável. Além disso, 
\begin{enumerate}
    \item $\displaystyle\left\|\int_a^b u(t)\,dt\right\|\leq \int_a^b \|u(t)\|\,dt$
    \item Se $A\in \mathcal{L}(X,Y)$, então
    \[ A\left(\int_a^b u(t)\,dt\right)=\int_a^b A(u(t))\,dt. \]
\end{enumerate}
\end{proposition}
\begin{proof}
Veja em \cite[Theorem A3.2]{kesavan2015topics}
\end{proof}

\begin{proposition}\label{Prop.VM}
Se $u:[a,a+h]\longrightarrow X$ é {contínua}, então
\begin{equation}\label{ineq.VM}
\lim\limits_{h\to 0^+} \frac{1}{h}\int_a^{a+h} u(t)\,dt = u(a).
\end{equation}
\end{proposition}
\begin{proof}
A função  $f:t\in [a,a+h]\longmapsto \|u(t)-u(a)\|\in \R$ é contínua. Pelo Teorema do Valor Médio para integrais, existe $\xi_h\in [a,a+h]$ tal que 
\[
\frac{1}{h}\int_a^{a+h} f(t)\,dt=f(\xi_h).
\]
Como $a\leq \xi_h\leq a+h$, se $h\to 0^+$, então $\xi_h\to a^+$. Neste caso, 
\[
\lim_{h\to 0^+}\frac{1}{h}\int_a^{a+h} f(t)\,dt=\lim_{h\to 0^+}f(\xi_h)=f(a).
\]
Isto é, 
\begin{equation*}
\lim_{h\to 0^+}\frac{1}{h}\int_a^{a+h} \|u(t)-u(a)\|\,dt=0.
\end{equation*}
Assim, 
\begin{align*}
\lim_{h\to 0^+}\left\|\frac{1}{h}\int_a^{a+h} u(t)\,dt -u(a)\right\|& =
\lim_{h\to 0^+}\left\|\frac{1}{h}\int_a^{a+h} u(t)-u(a)\,dt\right\|\\
&\leq \lim_{h\to 0^+}\frac{1}{h}\int_a^{a+h}\left\| u(t)-u(a)\right\|\,dt
=0.
\end{align*}
O que é equivalente à identidade \eqref{ineq.VM}.
\end{proof}