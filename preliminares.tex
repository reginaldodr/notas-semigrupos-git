\chapter{Preliminares}
\section{Espaços Normados}

\begin{definition}
    Uma \dt{norma} é \index{Norma} 
\end{definition}


\section{Operadores Lineares Ilimitados}

Sejam $X$ e $Y$ dois espaços de Banach. Seja $A:D(A)\subset X\longrightarrow Y$ um operador linear,  onde $D(A)$ é um subespaço de $X$, chamado de \dt{domínio}\index{Operador Linear!domínio} de $A$.
\medskip
Dizemos que $A$ é \dt{limitado (ou contínuo)} \index{Operador Linear!limtado} se $D(A)=X$ e se existe  $C>0$ tal que 
\begin{equation}\label{ltdo}
\|Ax\|_{Y}\leq C\|x\|_X, \ \forall x\in X.
\end{equation}
Denotaremos por $\mathcal{L}(X,Y)$ o espaço de Banach dos \dt{operadores lineares limitados} com norma dada por
\[\|A\|_{_{\mathcal{L}(X,Y)}}=\sup\limits_{x\in X, x\neq 0}\frac{\|Ax\|_{_Y}}{\|x\|_{_X}},\]
por $\mathcal{L}(X)=\mathcal{L}(X,X)$ e $X'=\mathcal{L}(X,\R)$.
$A$ é dito ser \dt{ilimitado} \index{Operador Linear!ilimitado} quando não satisfazer \eqref{ltdo}. Dizemos que $A$ é \dt{densamente definito} \index{Operador Linear!desamente definido} se $\overline{D(A)}=X$.

\begin{theorem}[Banach-Steinhaus]\label{th-BS}
	Sejam $X$ um espaço de Banach e $Y$ um espaço vetorial normado. Se $\{T_i\}_{i\in I}$  é uma família (não necessariamente enumerável) em $\mathcal{L}(X,Y)$ pontualmente limitada, então $\{T_i\}_{i\in I}$ é uniformemente limitada, isto é, se 
	 \[\sup_{i\in I}\|T_ix\|< \infty,\ \forall x\in X,\] 
	 então
	 \[\sup_{i\in I}\|T_i\|_{\mathcal{L}(X,Y)}<\infty.\]
\end{theorem}

\section{Integrais Vetoriais}

\begin{definition}
Sejam $X$ um espaço de Banach e $u:[a,b]\longrightarrow X$ uma aplicação tal que, para cada $\varphi\in X'$,  a função real
\[t\in [a,b] \longmapsto \langle\varphi, u(t) \rangle_{X'X}\in \R,\]
seja integrável. Dizemos que \dt{$u$ é integrável} se existe um vetor $v\in X$ que satisfaz:
\[\langle \varphi, v\rangle_{X',X}=\int_a^b \langle\varphi, u(t) \rangle_{X'X}\,dt,\ \forall \varphi\in X'.\]
Em caso afirmativo, $v$ é único e escrevemos
\[v=\int_a^b u(t)\,dt.\]
\end{definition}

\begin{proposition}\label{KthA3.2}
Se $u:[a,b]\longrightarrow X$ é {contínua}, então $u$ é integrável. Além disso, 
\begin{enumerate}
    \item $\displaystyle\left\|\int_a^b u(t)\,dt\right\|\leq \int_a^b \|u(t)\|\,dt$
    \item Se $A\in \mathcal{L}(X,Y)$, então
    \[ A\left(\int_a^b u(t)\,dt\right)=\int_a^b A(u(t))\,dt. \]
\end{enumerate}
\end{proposition}
\begin{proof}
Veja em \cite[Theorem A3.2]{kesavan2015topics}
\end{proof}

\begin{proposition}
Se $u:[a,a+h]\longrightarrow X$ é {contínua}, então
\begin{equation}\label{ineq.VM}
\lim\limits_{h\to 0^+} \frac{1}{h}\int_a^{a+h} u(t)\,dt = u(a).
\end{equation}
\end{proposition}
\begin{proof}
A função  $f:t\in [a,a+h]\longmapsto \|u(t)-u(a)\|\in \R$ é contínua. Pelo Teorema do Valor Médio para integrais, existe $\xi_h\in [a,a+h]$ tal que 
\[
\frac{1}{h}\int_a^{a+h} f(t)\,dt=f(\xi_h).
\]
Como $a\leq \xi_h\leq a+h$, se $h\to 0^+$, então $\xi_h\to a^+$. Neste caso, 
\[
\lim_{h\to 0^+}\frac{1}{h}\int_a^{a+h} f(t)\,dt=\lim_{h\to 0^+}f(\xi_h)=f(a).
\]
Isto é, 
\begin{equation*}
\lim_{h\to 0^+}\frac{1}{h}\int_a^{a+h} \|u(t)-u(a)\|\,dt=0.
\end{equation*}
Assim, 
\begin{align*}
\lim_{h\to 0^+}\left\|\frac{1}{h}\int_a^{a+h} u(t)\,dt -u(a)\right\|& =
\lim_{h\to 0^+}\left\|\frac{1}{h}\int_a^{a+h} u(t)-u(a)\,dt\right\|\\
&\leq \lim_{h\to 0^+}\frac{1}{h}\int_a^{a+h}\left\| u(t)-u(a)\right\|\,dt
=0.
\end{align*}
O que é equivalente à identidade \eqref{ineq.VM}.
\end{proof}