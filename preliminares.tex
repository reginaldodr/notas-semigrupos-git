%\chapter{Preliminares}

\chapter{Noções Básicas sobre Espaços de Banach}

\section{Espaços Normados}

No decorrer do presente capítulo, denotaremos por $\mathbb K$ o corpo dos números reais ou o corpo dos números complexos. Além disso, quando $(M,d)$ for um espaço métrico, cada bola aberta, cada bola fechada e cada esfera em $M$ serão denotadas por
\[
\displaystyle B(a;r)=\{y\in M; d(y,a)<r\},
\]

\[
\displaystyle B(a;r)=\{y\in M; d(y,a)\leq r\}
\]
e
\[
\displaystyle B(a;r)=\{y\in M; d(y,a)=r\},
\]
respectivamente, onde $a\in M$ e $r>0$.

\begin{definition}\label{norma}
    Seja $X$ um $\mathbb K$-espaço vetorial. Uma aplicação
    \[
    \displaystyle \| \cdot \|: X\longrightarrow \R
    \]
    é dita uma \dt{norma} \index{norma} se, para quaisquer $x,y\in X$ e $\lambda \in \mathbb K$, as seguintes condições se verificarem:
    \begin{itemize}
    \item[(a)] $\|x\|\geq 0$;
    \item[(b)] Se $\|x\|=0$, então $x=0$;
    \item[(c)] $\|\lambda x\|=|\lambda| \|x\|$;
    \item[(d)] $\|x+y\|\leq \|x\|+\|y\|$.
    \end{itemize}
    Nesse caso, o par $(X,\|\cdot\|)$ é dito um \dt{espaço normado} \index{Espaço!normado}.
\end{definition}

\begin{remark}
Em um espaço normado $(X,\|\cdot\|)$, valem:
    \begin{itemize}
    \item[(a)] $\|0\|=0$;
    \item[(b)] $|\|x\|-\|y\|| \leq \|x-y\|$ para quaisquer $x,y \in X$.
    \end{itemize}
\end{remark}

\begin{example}\label{ex21}
    Dado um número inteiro positivo $n$, não é difícil verificar que
    \[
    \displaystyle \|\cdot \|_0 : (x_1,\cdots , x_n) \in \mathbb K^n \longmapsto \bigg( \sum_{j=1}^{n} |x_j|^2\bigg)^{1/2} \in \R ,
    \]
    \[
    \displaystyle \|\cdot \|_1 : (x_1,\cdots , x_n) \in \mathbb K^n \longmapsto \max_{1\leq j \leq n} |x_j| \in \R 
    \]
    e
    \[
    \displaystyle \|\cdot \|_2 : (x_1,\cdots , x_n) \in \mathbb K^n \longmapsto \sum_{j=1}^{n} |x_j| \in \R 
    \]
    são normas em $\mathbb K ^n$.
\end{example}

\begin{example}\label{limitadas}
Seja $A$ um conjunto não vazio. Denotemos por $\mathcal B (A)$ o conjunto de todas as funções limitadas $f: A\longrightarrow \mathbb K$. Dados $f,g \in \mathcal B (A)$ e $\lambda \in \mathbb K$, definamos
\begin{itemize}
\item $(f+g)(x):=f(x)+g(x)$; 
\item $(\lambda f)(x) = \lambda f(x)$,
para cada $x\in A$.
\end{itemize}
Com as operações operações de adição e multiplicação por escalar, pontualmente dadas acima, $\mathcal B (A)$ é um $\mathbb K$-espaço vetorial. Além disso, não é difícil constatar que
\[
\displaystyle f\in \mathcal B (A) \longmapsto \| f\| = \sup_{x\in A} |f(x)| \in \R
\]
é uma norma em $\mathcal B (A)$. Observemos ainda que a convergência de sequências em $\mathcal B (A)$ é exatamente a noção de convergência uniforme. 
\end{example}

\begin{definition}
    Com as notações do Exemplo \ref{limitadas},
    \[
    \ell_{\infty} := \mathcal B (\mathbb N),
    \]
    que é o espaço normado de todas as sequências limitadas cujos termos pertencem a $\mathbb K$.
\end{definition}

\begin{example}\label{lp}
    Seja $p\in [1,+\infty )$ e definamos 
    \[
    \displaystyle \ell_p =\left\{ (x_n )_{n=1}^{\infty}; \sum_{n=1}^{\infty} |x_n|^p <+\infty \right\}
    \]
    
    Mostraremos que as operações 
    \[
    \bigg( (x_n )_{n=1}^{\infty},(y_n )_{n=1}^{\infty} \bigg) \in \ell_p \times \ell_p \longmapsto (x_n +y_n )_{n=1}^{\infty} \in \ell_p
    \]
    e
    \[
    \bigg( \lambda ,(x_n )_{n=1}^{\infty} \bigg) \in \mathbb K \times \ell_p \longmapsto (\lambda x_n )_{n=1}^{\infty} \in \ell_p
    \]
    estão bem definidas e que a aplicação 
    \[
    \displaystyle (x_n )_{n=1}^{\infty} \in \ell_p \longmapsto \bigg( \sum_{n=1}^{\infty} |x_n|^p \bigg)^{1/p} \in \R .
    \]
    Tais objetivos serão alcançados no Corolário \ref{dms}.
\end{example}

\begin{proposition}[Desigualdade de Hölder para somas] \label{dhsomas}
    Sejam $p,q \in (1,+\infty)$ tais que 
    \[
    \displaystyle \frac{1}{p} + \frac{1}{q} =1.
    \]
    Se $(x_1, \cdots , x_n) \in \mathbb K^n$ e $(y_1, \cdots , y_n) \in \mathbb K^n$, então 
    \[
    \displaystyle \sum_{j=1}^{n} |x_j y_j|\leq \bigg( \sum_{j=1}^{n} |x_j|^p \bigg)^{1/p} \bigg( \sum_{j=1}^{n} |y_j|^q \bigg)^{1/q} .
    \]
\end{proposition}
\begin{proof}
    Em virtude do Teorema do Valor Médio, não é difícil constatar que, para quaisquer $a,b\in [0,+\infty )$ e $\alpha \in [0,1]$, tem-se
    \begin{equation}\label{TVM}
    \displaystyle a^{\alpha} b^{1-\alpha} \leq \alpha a + (1-\alpha )b.
    \end{equation}
    Como claramente podemos supor que
    \[
    \sum_{j=1}^{n} |x_j|^p >0 \text{  e  } \sum_{j=1}^{n} |y_j|^p >0,
    \]
    consideremos 
    \[
    \displaystyle a_{m} = \frac{|x_{m}|^p}{\sum_{j=1}^{n} |x_j|^p} \text{  e  } b_{m} = \frac{|y_{m}|^q}{\sum_{j=1}^{n} |y_j|^p},
    \]
    para cada $m\in \{1,\cdots , n\}$. Logo, tomando $\alpha = \frac{1}{p}$ e aplicando \eqref{TVM}, temos
    \[
    \displaystyle a_{m}^{\alpha} b_{m}^{1-\alpha} \leq \alpha a_m + (1-\alpha )b_m ,
    \]
    isto é,
    \begin{equation}\label{ineq}
    \displaystyle \frac{|x_{m}|}{\bigg(\sum_{j=1}^{n} |x_j|^p \bigg)^{1/p}} \cdot \frac{|y_{m}|}{\bigg(\sum_{j=1}^{n} |y_j|^p \bigg)^{1/q}}
    \leq \frac{1}{p} \cdot \frac{|x_{m}|^p}{\sum_{j=1}^{n} |x_j|^p}+ \frac{1}{q} \cdot \frac{|y_{m}|^q}{\sum_{j=1}^{n} |y_j|^p},
    \end{equation}
    para cada $m\in \{ 1,\cdots , n\}$. Somando membro a membro as $n$ relações descritas em \eqref{ineq}, fica demonstrada a desigualdade de Hölder do enunciado.
\end{proof}

\begin{corollary}[Desigualdade de Hölder para séries]
    Sejam $p,q \in (1,+\infty)$ satisfazendo $\frac{1}{p} + \frac{1}{q} =1$. Se $(x_n)_{n=1}^{\infty} \in \ell _p$ e $(y_n)_{n=1}^{\infty} \in \ell _q$, então $(x_n y_n)_{n=1}^{\infty} \in \ell _1$ e vale
    \[
    \displaystyle \sum_{n=1}^{\infty} |x_n y_n|\leq \bigg( \sum_{n=1}^{\infty} |x_n|^p \bigg)^{1/p} \bigg( \sum_{n=1}^{\infty} |y_n|^q \bigg)^{1/q} .
    \]
\end{corollary}
\begin{proof}
Segue imediatamente da Proposição \ref{dhsomas}.
\end{proof}


\begin{proposition}[Desigualdade de Minkowski para somas]
Seja $p\in (1,+\infty)$. Se $(x_1, \cdots , x_n) \in \mathbb K^n$ e $(y_1, \cdots , y_n) \in \mathbb K^n$, então 
    \[
    \displaystyle \bigg( \sum_{j=1}^{n} |x_j + y_j|^p \bigg)^{1/p} \leq \bigg( \sum_{j=1}^{n} |x_j|^p \bigg)^{1/p} +\bigg( \sum_{j=1}^{n} |y_j|^p \bigg)^{1/p} .
    \]
\end{proposition}

\begin{proof}
Como claramente podemos supor que
\[
\displaystyle \sum_{j=1}^{n} |x_j + y_j|^p >0,
\]
a Proposição \ref{dhsomas} nos dá

\begin{align}
\displaystyle \sum_{j=1}^{n} |x_j + y_j|^p
&=\sum_{j=1}^{n} |x_j + y_j|^{p-1} |x_j + y_j| \nonumber \\
&\leq \sum_{j=1}^{n} |x_j + y_j|^{p-1} |x_j|
+\sum_{j=1}^{n} |x_j + y_j|^{p-1} |y_j| \nonumber \\
&\leq \bigg( \sum_{j=1}^{n} |x_j + y_j|^{(p-1)q} \bigg)^{1/q} \bigg( \sum_{j=1}^{n} |x_j|^p \bigg)^{1/p} \nonumber\\
&\hspace{1cm}+\bigg( \sum_{j=1}^{n} |x_j + y_j|^{(p-1)q} \bigg)^{1/q} \bigg( \sum_{j=1}^{n} |y_j|^p \bigg)^{1/p} \nonumber \\
&=\bigg( \sum_{j=1}^{n} |x_j + y_j|^{p} \bigg)^{1/q} \bigg[ \bigg( \sum_{j=1}^{n} |x_j|^p \bigg)^{1/p} + \bigg( \sum_{j=1}^{n} |y_j|^p \bigg)^{1/p}\bigg] \label{ineqm},
\end{align}
onde $q=\frac{p}{p-1}$. Multiplicando os membros de \eqref{ineqm} por $\bigg( \sum_{j=1}^{n} |x_j + y_j|^p \bigg)^{-1/q}$, fica demonstrada a desigualdade de Minkowski do enunciado.
\end{proof}

\begin{corollary}[Desigualdade de Minkowski para séries]\label{dms}
    Seja $p\in (1,+\infty)$. Se $x=(x_n)_{n=1}^{\infty} \in \ell_p $ e $y=(y_n)_{n=1}^{\infty} \in \ell_q$, então $(x_n + y_n)_{n=1}^{\infty} \in \ell_p$ e vale
     \[
    \displaystyle \bigg( \sum_{n=1}^{\infty} |x_n + y_n|^p \bigg)^{1/p} \leq \bigg( \sum_{n=1}^{\infty} |x_n|^p \bigg)^{1/p} +\bigg( \sum_{n=1}^{\infty} |y_n|^p \bigg)^{1/p} .
    \]
    Em outras palavras, $x+y \in \ell_p$, ou seja, a adição em $\ell_p$ apresentada no Exemplo \ref{lp} encontra-se bem definida e, além disso, vale a desigualdade triangular
    \[
    \displaystyle \|x+y\|_p \leq \|x\|_p +\|y\|_p.
    \]
    Assim, sendo imediata a boa definição da multiplicação por escalar em $\ell_p$, também declarada no Exemplo \ref{lp}, resulta que $(\ell_p , \|\cdot\|_p)$ é um espaço vetorial normado.
\end{corollary}    

\begin{example}\label{ex23}
    Sendo $M$ um espaço métrico, denotemos por $\mathcal C_b (M)$ o subconjunto de $\mathcal B (M)$ formado por todas as funções contínuas e limitadas de $M$ em $\mathbb K$. Não é difícil constatar que $\mathcal C_b (M)$ é um subespaço vetorial fechado de $\mathcal B (M)$.
\end{example}

\begin{example}\label{ex24}
Denotemos por
\begin{itemize}
\item $\textbf{c}$ o conjunto de todas as sequências convergentes em $\mathbb K$;
\item $\textbf{c}_0$ o conjunto de todas as sequências convergentes em $\mathbb K$ convergindo para zero;
\item $\textbf{c}_{00}$ o conjunto de todas as sequências $(x_n)_{n=1}^{\infty}$ em $\mathbb K$ com a seguinte propriedade: existe um inteiro positivo $n_0$ tal que $x_n=0$ para todo $n\geq n_0$.
\end{itemize}
Claramente, 
\[
\textbf{c}_{00} \subset \textbf{c}_{0} \subset \textbf{c} \subset \ell_{\infty}.
\]
Além disso, não é difícil constatar que $\textbf{c}$ é um subespaço vetorial fechado de $\ell_{\infty}$.
\end{example}

O próximo resultado traz importantes caracterizações das aplicações lineares contínuas.

\begin{proposition}\label{continuous}
Sejam $X$ e $Y$ dois espaços normados, e consideremos uma aplicação linear $T:X\longrightarrow Y$. As seguintes condições são equivalentes:

\begin{itemize}
\item[(a)] $T$ é uniformemente contínua;
\item[(b)] $T$ é contínua;
\item[(c)] $T$ é contínua em $0\in X$;
\item[(d)] Existe $C>0$ tal que $\|T(x)\|_Y \leq C \|x\|_X$ para todo $x\in X$.
\end{itemize}
\end{proposition}

\begin{proof}
É claro que $(a)\Longrightarrow (b) \Longrightarrow (c)$. Para vermos que $(c)\Longrightarrow (d)$, , tomemos $\varepsilon >0$. Como $T$ é contínua em $0\in X$, existe $\delta >0$ tal que
\[
\displaystyle \|T(x)\|_Y =\|T(x) - T(0)\|_Y <\varepsilon =1,
\]
sempre que $x\in X$ e $\|x\|_X = \|x-0\|_X <\delta$. Assim, para todo $w\in X\setminus \{0\}$, temos
\[
\displaystyle \bigg\|T\bigg( \frac{\delta}{2} \bigg( \frac{w}{\|w\|_X} \bigg)\bigg) \bigg\|_Y <1.
\]
Logo, 
\[
\|T(w)\|_Y \leq \frac{2}{\delta}  \|w\|_X \text{  para todo  } w\in X,
\]
inclusive se $w=0$.

Agora, para vermos que $(d)\Longrightarrow (a)$, basta observarmos que
\[
\|T(x) - T(y)\|_Y =\|T(x-y)\|_Y \leq C \|x-y\|_X
\]
para quaisquer $x,y\in X$. Ou seja: $T$ é uma aplicação lipschitziana e, portanto, uniformentente contínua.
\end{proof}

\begin{definition}
    Sejam $X$ e $Y$ dois espaços normados. Denotaremos por $\mathcal L (X,Y)$ o espaço vetorial de todas as transformações lineares de $X$ em $Y$, com as operações de adição e multiplicação por escalar definidas pontualmente. Cabe observar que:
    \begin{itemize}
    \item[(a)] Quando $X=Y$, escreveremos $\mathcal L (X)$ em vez de $\mathcal L (X,X)$;
    \item[(b)] Quando $Y=\mathbb K$, denotaremos $\mathcal L (X,\mathbb K)$ por $X'$, que é conhecido como \dt{o dual topológico} \index{dual topológico} de X;
    \item[(c)] O conjunto de todos os funcionais lineares $\varphi : X\longrightarrow \mathbb K$, contínuos ou não, será denotador por $X^{\ast}$, que é conhecido como \dt{o dual algébrico} \index{dual algébrico} de X.
    \end{itemize}
\end{definition}

\begin{remark}
Seja $X$ um espaço normado. É sempre verdade que $X' \subset X^{\ast}$. Além disso, quando $X$ tem dimensão infinita, sempre temos $X'\neq X^{\ast}$. De fato, em virtude do Lema de Zorn, sabemos que $X$ possui uma base (de Hamel)
\[
\displaystyle \mathcal B = \{ x_i; i\in I\},
\]
onde $I$ é um conjunto infinito. Podemos supor que $\| x_i \|_X =1$ para todo $i\in I$. Agora, seja 
\[
\displaystyle J = \{ i_1, i_2, \cdots , i_k, \cdots \}
\]
um subconjunto infinito e enumerável de $I$. Logo, existe um único funcional linear $\varphi : X \longrightarrow \mathbb K$ tal que 
\begin{itemize}
\item $\varphi (x_{i_k})=k$ para cada $k\in \mathbb N$;
\item $\varphi (x_i)=0$ se $i\in I\setminus J$.
\end{itemize}
Um vez que 
\[
\displaystyle \mathbb N \subset \left\{ \frac{|\varphi (x)|}{\|x\|_{X}}; x\in X\setminus \{0\}\right\},
\]
não existe $C>0$ de modo que valha $|\varphi (x)|\leq \|x\|_X$ para todo $x\in X$. Pela Proposição \ref{continuous}, $\varphi$ não é contínua, isto é, $\varphi \in X^{\ast} \setminus X'$.
\end{remark}

\begin{remark}
Sejam $X$ e $Y$ dois espaços normados. Não é difícil constatar que:
\begin{itemize}
\item[(a)] Se $X$ tem dimensão finita, então toda aplicação linear $T:X\longrightarrow Y$ é contínua;
\item[(b)] Se X tem dimensão infinita, sempre existe uma aplicação linear $T:X\longrightarrow Y$ que não é contínua.
\end{itemize}
\end{remark}

\begin{definition}
Sejam $X$ e $Y$ dois espaços normados. Uma aplicação linear $T:X 
\longrightarrow Y$ é dita \dt{limitada} \index{Operador Linear!limtado} 
se 
\begin{equation}\label{ltdo}
\sup_{x\in X\setminus \{0\}} \frac{\|T(x)\|_Y}{\|x\|_X} 
<\infty.
\end{equation}

\end{definition}

\begin{remark}
Sejam $X$ e $Y$ dois espaços normados. Em virtude da Proposição 
\ref{continuous}, uma aplicação linear $T:X\longrightarrow Y$ é contínua 
se, e somente se, é limitada. 
\end{remark}

\begin{example}
Sejam $X$ e $Y$ dois espaços normados. Vejamos que 
\[
T\in \mathcal L (X,Y) \longmapsto \|T\|_{\mathcal L (X,Y)} = \sup_{x\in X\setminus \{0\}} \frac{\|T(x)\|_Y}{\|x\|_X} \in \R
\]
é uma norma em $\mathcal L (X,Y)$. De fato, as condições (a), (b) e (c) da Definição \ref{norma} são claras. Além disso, se $T,S\in \mathcal L (X,Y)$, temos

\begin{equation*}
\displaystyle \|(T+S)(x)\|_Y=\|T(x)+S(x)\|_Y \leq \|T(x)\|_Y +\|S(x)\|_Y \leq (\|T\|_{\mathcal L (X,Y)} + \|S\|_{\mathcal L (X,Y)})\|x\|_X
\end{equation*}
para todo $x\in E$, ou seja,
\begin{equation*}
    \displaystyle \|T+S\|_{\mathcal L (X,Y)}=\sup_{x\in X\setminus \{0\}} \frac{\|(T+S)(x)\|_Y}{\|x\|_X} \leq \|T\|_{\mathcal L (X,Y)}+\|S\|_{\mathcal L (X,Y)}.
\end{equation*}
Assim, temos realmente uma norma em $\mathcal L (X,Y)$.
\end{example}

\begin{definition}
Sejam $X$ e $Y$ dois espaços normados. Uma aplicação linear $T:X\longrightarrow Y$ é dita um \dt{isomorfismo topológico} \index{isomorfismo topológico} se for um homeomorfismo. 
\end{definition}

\begin{definition}
Sejam $\|\cdot\|_1$ e $\|\cdot\|_2$ duas normas em um mesmo espaço vetorial $X$. Dizemos que tais normas são \dt{equivalentes} \index{norma!equivalentes} se a aplicação identidade
\[
\displaystyle I_X : (X,\|\cdot\|_1) \longrightarrow (X,\|\cdot\|_2)
\]
for um isomorfismo topológico.
\end{definition}

\begin{corollary}
Sejam $\|\cdot\|_1$ e $\|\cdot\|_2$ duas normas em um mesmo espaço vetorial $X$. As seguintes condições são equivalentes:
\begin{itemize}
\item[(a)] $\|\cdot\|_1$ e $\|\cdot\|_2$ são equivalentes;
\item[(b)] Existem constantes $A>0$ e $B>0$ tais que
\[
A\|x\|_2 \leq \|x\|_1 \leq \|x\|_2
\]
para todo $x\in X$.
\end{itemize}
\end{corollary}

\begin{proof}
Basta aplicar a Proposição \ref{continuous}.
\end{proof}

\section{Espaços de Banach}

\begin{definition}
Seja $X$ um espaço normado. Uma sequência $(x_n)_{n\in \mathbb N}$ em $X$ é dita ser uma \dt{sequência de Cauchy} \index{Sequência de Cauchy} quando para todo $\ep>0$ dado, existe um número $n_0\in \mathbb N$ tal que 
\begin{equation*}
    m,n>n_0 \Rightarrow \|x_m-x_n\|< \ep.
\end{equation*}
Dizemos que $X$ é um \dt{espaço de Banach} \index{Espaço! de Banach} se ele for completo, isto é, quando toda sequência de Cauchy for  convergente.
\end{definition}

\begin{remark}\label{subespaço}
Se $X$ é um espaço métrico, então cada subespaço fechado $M$ de $X$ é também um espaço de Banach, com a norma induzida pela norma de $X$.
\end{remark}

\begin{example}
Dado um inteiro positivo $n$, $\mathbb K ^n$ é um espaço de Banach com a norma $\|\cdot\|_0$ do Exemplo \ref{ex21}. Na verdade, como as normas $\| \cdot\|_0$, $\| \cdot\|_1$ e $\| \cdot\|_2$ são duas a duas equivalentes, $(\mathbb K ^n ,\| \cdot\|_j )$ é um espaço de Banach para todo $j\in \{1,2,3\}$. 
\end{example}

\begin{example}
Dado um conjunto $A\neq \emptyset$, não  é difícil constatar que $\mathcal B (A)$, introduzido no Exemplo \ref{limitadas}, é um espaço de Banach. Em particular, $\ell _{\infty}$ é um espaço de Banach.
\end{example}

\begin{example}
    Para cada espaço métrico $M$, $\mathcal C _b (M)$, apresentado no \ref{ex23} é um espaço de Banach, uma vez que é um subespaço fechado de $\mathcal B (M)$ (veja a Observação \ref{subespaço}).
\end{example}

\begin{example}
O subespaço vetorial $\textbf{c}$ de $\ell_{\infty}$, mencionado no Exemplo \ref{ex24}, é um espaço de Banach, em virtude da Proposição \ref{continuous} e da Observação \ref{subespaço}.
\end{example}

\begin{example}
O subespaço de $\textbf{c}_{00}$ de $\ell_{\infty}$, mencionado no Exemplo \ref{ex24}, NÃO é um espaço de Banach. De fato, para cada inteiro $n \geq 1$, consideremos
\[
\displaystyle x_n =\bigg( 1,\frac{1}{2}, \frac{1}{3}, \cdots , \frac{1}{n}, 0, 0, 0,\cdots \bigg) \in \textbf{c}_{00}.
\]
É fácil ver que $(x_n)_{n=1}^{\infty}$ é uma sequência em $\textbf{c} _{00}$ que converge a 
\[
\displaystyle x =\bigg( 1,\frac{1}{2}, \frac{1}{3}, \cdots , \frac{1}{n}, \cdots \bigg) \in \ell_{\infty} \setminus\textbf{c}_{00}.
\]
Isto significa que $(x_n)_{n=1}^{\infty}$ é uma sequência de Cauchy em $\textbf{c}_{00}$ que NÃO CONVERGE em $\textbf{c}_{00}$.
\end{example}

\begin{remark}
    Aqui enfatizamos algumas observações importantes sobre espaços de Banach:
\begin{itemize}
\item[(a)] Sendo $X$ e $Y$ dois espaços normados, para que $\mathcal L (X,Y)$ seja um espaço de Banach, é necessário e suficiente que $Y$ seja completo.
\item[(b)] Para cada $p\in [1,\infty )$, $\ell_p$ é um espaço de Banach.
\item[(c)] Para que um espaço normado $X$ seja completo, é necessário e suficiente que toda série absolutamente convergente em $X$ seja convergente.
\end{itemize}
\end{remark}

\section{Espaços normados de dimensão finita}

\section{Operadores Lineares Ilimitados}

Sejam $X$ e $Y$ dois espaços de Banach. Seja $A:D(A)\subset X\longrightarrow Y$ um operador linear,  onde $D(A)$ é um subespaço de $X$, chamado de \dt{domínio}\index{Operador Linear!domínio} de $A$.
%Dizemos que $A$ é \dt{limitado (ou contínuo)}  se $D(A)=X$ e se existe  $C>0$ tal que 
%\begin{equation}\label{ltdo}
%\|Ax\|_{Y}\leq C\|x\|_X, \ \forall x\in X.
%\end{equation}
%Denotaremos por $\mathcal{L}(X,Y)$ o espaço de Banach dos \dt{operadores lineares limitados} com norma dada por
%\[\|A\|_{_{\mathcal{L}(X,Y)}}=\sup\limits_{x\in X, x\neq 0}\frac{\|Ax\|_{_Y}}{\|x\|_{_X}},\]
%por $\mathcal{L}(X)=\mathcal{L}(X,X)$ e $X'=\mathcal{L}(X,\R)$.
$A$ é dito ser \dt{ilimitado} \index{Operador Linear!ilimitado} quando não satisfazer \eqref{ltdo}. Dizemos que $A$ é \dt{densamente definito} \index{Operador Linear!desamente definido} se $\overline{D(A)}=X$.

\begin{theorem}[Banach-Steinhaus]\label{th-BS}
	Sejam $X$ um espaço de Banach e $Y$ um espaço vetorial normado. Se $\{T_i\}_{i\in I}$  é uma família (não necessariamente enumerável) em $\mathcal{L}(X,Y)$ pontualmente limitada, então $\{T_i\}_{i\in I}$ é uniformemente limitada, isto é, se 
	 \[\sup_{i\in I}\|T_ix\|< \infty,\ \forall x\in X,\] 
	 então
	 \[\sup_{i\in I}\|T_i\|_{\mathcal{L}(X,Y)}<\infty.\]
\end{theorem}

\section{Integrais Vetoriais}

\begin{definition}
Sejam $X$ um espaço de Banach e $u:[a,b]\longrightarrow X$ uma aplicação tal que, para cada $\varphi\in X'$,  a função real
\[t\in [a,b] \longmapsto \langle\varphi, u(t) \rangle_{X'X}\in \R,\]
seja integrável. Dizemos que \dt{$u$ é integrável} se existe um vetor $v\in X$ que satisfaz:
\[\langle \varphi, v\rangle_{X',X}=\int_a^b \langle\varphi, u(t) \rangle_{X'X}\,dt,\ \forall \varphi\in X'.\]
Em caso afirmativo, $v$ é único e escrevemos
\[v=\int_a^b u(t)\,dt.\]
\end{definition}

\begin{proposition}\label{KthA3.2}
Se $u:[a,b]\longrightarrow X$ é {contínua}, então $u$ é integrável. Além disso, 
\begin{enumerate}
    \item $\displaystyle\left\|\int_a^b u(t)\,dt\right\|\leq \int_a^b \|u(t)\|\,dt$
    \item Se $A\in \mathcal{L}(X,Y)$, então
    \[ A\left(\int_a^b u(t)\,dt\right)=\int_a^b A(u(t))\,dt. \]
\end{enumerate}
\end{proposition}
\begin{proof}
Veja em \cite[Theorem A3.2]{kesavan2015topics}
\end{proof}

\begin{proposition}
Se $u:[a,a+h]\longrightarrow X$ é {contínua}, então
\begin{equation}\label{ineq.VM}
\lim\limits_{h\to 0^+} \frac{1}{h}\int_a^{a+h} u(t)\,dt = u(a).
\end{equation}
\end{proposition}
\begin{proof}
A função  $f:t\in [a,a+h]\longmapsto \|u(t)-u(a)\|\in \R$ é contínua. Pelo Teorema do Valor Médio para integrais, existe $\xi_h\in [a,a+h]$ tal que 
\[
\frac{1}{h}\int_a^{a+h} f(t)\,dt=f(\xi_h).
\]
Como $a\leq \xi_h\leq a+h$, se $h\to 0^+$, então $\xi_h\to a^+$. Neste caso, 
\[
\lim_{h\to 0^+}\frac{1}{h}\int_a^{a+h} f(t)\,dt=\lim_{h\to 0^+}f(\xi_h)=f(a).
\]
Isto é, 
\begin{equation*}
\lim_{h\to 0^+}\frac{1}{h}\int_a^{a+h} \|u(t)-u(a)\|\,dt=0.
\end{equation*}
Assim, 
\begin{align*}
\lim_{h\to 0^+}\left\|\frac{1}{h}\int_a^{a+h} u(t)\,dt -u(a)\right\|& =
\lim_{h\to 0^+}\left\|\frac{1}{h}\int_a^{a+h} u(t)-u(a)\,dt\right\|\\
&\leq \lim_{h\to 0^+}\frac{1}{h}\int_a^{a+h}\left\| u(t)-u(a)\right\|\,dt
=0.
\end{align*}
O que é equivalente à identidade \eqref{ineq.VM}.
\end{proof}