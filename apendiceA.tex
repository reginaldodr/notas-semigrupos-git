                                 
                            \chapter{Apêndice}

\section{Resultados Clássicos}


\begin{proposition}[Teste de Weierstrass]\label{TWeierstrass}
Seja $f:[a,+\infty)\times \Lambda\longrightarrow X$, $\Lambda$ 
$\overbrace{\text{um subconjunto aberto de }\mathbb{C}}^{\text{Não seria } \R\,?}$ 
 contínua em $t\in [a,+\infty)$ para cada $\lambda \in \Lambda$ Se existe $M:[a,+\infty)\longrightarrow \R$ contínua e positiva em $t\in [a,+\infty)$ tal que
\begin{enumerate}[$(i)$]
\item $\|f(t,\lambda)\|\leq M(t),\quad \forall (t,\lambda)\in [a,+\infty)\times \Lambda$,
\item $\displaystyle \int_a^\infty M(t)\, dt<+\infty$.
\end{enumerate} 
Então 
\[\int_a^\infty f(t,\lambda)\, dt\]
converge absolutamente para cada $\lambda $ pertecente ao conjunto $\Lambda$ e a convergência é uniforme nesse conjunto.
\end{proposition}
\begin{proof}
{\color{red}CITAR}
\end{proof}


A seguir, enunciamos dois resultados básicos da Análise Funcional que serão utilizados nas exposições deste minicurso. O primeiro deles é o Princípio da Limitação Uniforme (versão do Teorema de Banach-Steinhaus no contexto dos espaços normados) e o outro é o Teorema do Gráfico Fechado, ambos consequências do importante Lema de Baire (veja \cite{pombo1999introduccao}).

\begin{theorem}[Banach-Steinhaus]\label{th-BS}
	Sejam $X$ e $Y$ dois espaços normado, com $X$ completo, e consideremos uma família
    \[
    \displaystyle \mathcal F =\{T_i :X\longrightarrow Y; i\in I\} \subset \mathcal L (X,Y),
    \]
    não necessariamente enumerável, com a seguinte propriedade: para cada $x\in X$, temos
    \[
    \sup_{i\in I}\|T_ix\|_{Y} < \infty .
    \] 
    Então 
    \[
    \sup_{i\in I}\|T_i\|_{\mathcal{L}(X,Y)}<\infty.
    \]
    Em outras palavras, a limitação pontual da família $\mathcal F$ implica a sua limitação uniforme. 
    %Se $\{T_i\}_{i\in I}$  é uma família (não necessariamente enumerável) em $\mathcal{L}(X,Y)$ pontualmente limitada, então $\{T_i\}_{i\in I}$ é uniformemente limitada, isto é, se 
	% \[\sup_{i\in I}\|T_ix\|< \infty,\ \forall x\in X,\] 
%	 então
	% \[\sup_{i\in I}\|T_i\|_{\mathcal{L}(X,Y)}<\infty.\]
\end{theorem}

\begin{remark}
\begin{itemize}
\item[(a)] No Teorema \ref{th-BS}, a hipótese de que $X$ seja coompleto não pode ser removida. De fato, definindo 
\[
\displaystyle \varphi_n : x=(x_j)_{j=1}^{\infty} \in c_{00} \longmapsto nx_n \in \mathbb K ,
\]
para cada inteiro positivo $n$, fica estabelecida uma família pontualmente limitada que não é uniformemente limitada (relembre o Exemplo \ref{incompleto} e consulte \cite{pombo1999introduccao}).
\item[(b)] Ainda com as notações do Teorema de \ref{th-BS}, a limitação uniforme de $\mathcal F$, expressa por 
\[
\sup_{i\in I}\|T_i\|_{\mathcal{L}(X,Y)}<\infty ,
\]
é o mesmo que dizer que $\mathcal F$ é uma família equicontínua (de fato, basta adaptar a demonstração da Proposição \ref{continuous} para obter este fato).
\end{itemize}
\end{remark}

\begin{theorem}[Gráfico Fechado]\label{th-GF}
    Sejam $X$ e $Y$ dois espaços de Banach, e consideremos uma aplicação linear $T:X\longrightarrow Y$. Se
    \[
    \displaystyle G_T = \{ (x,y)\in X\times Y; y=Tx\}
    \]
    é um subespaço fechado de $X\times Y$, então $T$ é contínua.
\end{theorem}

\begin{remark}
    \begin{itemize}
    \item[(a)] Dentre as hipóteses do Teorema \ref{th-GF}, a completude de cada um dos dois espaços não pode ser removida (veja \cite{pombo1999introduccao}).
    \item[(b)] Conforme o enunciado do Teorema do \ref{th-GF}, $G_T$ é um subespaço vetorial de $X\times Y$ (não apenas um subconjunto do mesmo, como ocorre em geral para aplicações entre dois conjuntos dados). Isto é verdadeiro porque $T:X\longrightarrow Y$ é uma aplicação linear. 
    \item[(c)] É fácil ver que a aplicação linear $T:X\longrightarrow Y$ se exprime como a composição
    \[
    T=\pi_2 \circ (\pi_1 |_{G_T})^{-1},
    \]
    onde $\pi_1$ e $\pi_2$ são as projeções naturais de $X\times Y$ sobre $X$ e $Y$, respectivamente. A principal dificuldade na obtenção do Teorema \ref{th-GF} é demonstrar que $(\pi_1 |_{G_T})^{-1}: X\longrightarrow G_T$ é uma aplicação contínua (veja \cite{pombo1999introduccao}).
    \end{itemize}
\end{remark}

A seguir, apresentamos uma importante consideração sobre o operador resolvente mencionado na Definição \ref{res}.

\begin{remark}\label{reslim}
    Consideremos as notações da Definição \ref{res}. Para cada $\lambda \in \rho (A)$, vejamos que o operador linear $\rla :X\longrightarrow D(A) \subset X$ é limitado. Com efeito, como $X$ é um espaço de Banach, o Teorema do \ref{th-GF} garante que é suficiente verificarmos que o gráfico $G$ de $\rla$ é um subespaço fechado de $X\times X$. Nesse caso, consideremos uma sequência $(x_n ,\rla x_n )_{n=1}^{\infty}$ em $G$ convergindo a $(x,y)\in X\times X$. Isto significa que
    \begin{equation}\label{conv}
    \displaystyle x_n \to x \text{  e 
 } \rla x_n \to y \text{ em } X.
    \end{equation}
    Da Observação \ref{rl} (a), para todo $n\in \mathbb N$, temos
    \[
    \displaystyle A\rla x_n = \lambda \rla x_n -x_n 
    %\Longrightarrow x_n = \lambda \rla x_n - A\rla x_n \in D(A).
    \]
    %Portanto, pela Observação \ref{rl}, os operadores $A$ e $\rla$ comutam em cada termo da sequência $(x_n )_{n=1}^{\infty}$, o que implica
    %\[
    %\displaystyle x_n = \lambda \rla %x_n - \rla A x_n 
    %\]
    %para todo $n\in \mathbb N$. 
    Das convergências em \eqref{conv}, temos
    \[
    \displaystyle\lim_{n\to +\infty} A\rla x_n 
    %=\lim_{n\to +\infty} \rla Ax_n 
    = \lim_{n\to +\infty} [\lambda \rla x_n - x_n] = \lambda y-x \text{  em  } X.
    \]
    Como 
    \[
    \displaystyle y_n :=\rla x_n \to y \text{  e   } Ay_n  = A\rla x_n \to \lambda y-x
    \]
    em $X$, o fato de $A$ ser um operador fechado nos diz que $y\in D(A)$ e que $Ay = \lambda y-x$. Logo,
    \[
    \displaystyle (\lambda \id -A)y =x \Longrightarrow \rla x=y \Longrightarrow (x,y)\in G.
    \]
\end{remark}
